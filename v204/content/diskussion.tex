\section{Diskussion}
\label{sec:Diskussion}
Die Abbildung \ref{fig:statisch} zeigt zunächst, dass jedes der vier Metalle
zu Anfang schnell aufheizt doch danach langsam und linear in der Temperatur
anwächst. Da die Temperatur von Aluminium schneller wächst als die der anderen
Metalle, ist Aluminium die höchste Wärmeleitfähigkeit $κ$ zuzuordnen.
Dies lässt sich bereits aus Tabelle \ref{tab:mateig} entnehmen. Bei Aluminium
ist außerdem die Änderung der Temperatur bezogen auf den Ort größer, weshalb
auch der Wärmestrom größer sein muss. In Tabelle \ref{tab:waermestrom} wird das
ebenfalls verdeutlicht. Die berechneten Wärmeströme sind für Aluminium höher
als für Messing und Edelstahl. Die Kurve des schmalen Messingsstabes liegt in
der Abbildung \ref{fig:statisch} nahe an der breiten Messing-Kurve, die Werte
der Tabelle zeigen jedoch, dass nur der Faktor $\sfrac{12}{7}$ der Unterschied
zu den breiten Werten ist. Die Wärmeströme werden nicht 0, da an den Enden der
Stäbe keine Isolierung anliegt und die Isolierung sonst nicht komplett abschirmt.

Bei den einzelnen Werten der Metalle fällt auf,
dass nach ein paar Perioden die berechneten Größen konstant sind.
Diese liegen dann auch näher an den Theoriewerten, als die Mittelwerte.
Zu diesen Abweichungen kann es durch Einschwingvorgänge kommen,
da es eine angeregte Welle innerhalb des Stabes ist.
Die berechneten Werte liegen unter den Literaturwerten, da die thermische Isolierung
nicht an allen Seiten der Stäbe anliegt und das System nicht ideal abschirmt.
Der größere Fehler ist jedoch die Bestimmung der Amplitude einer Periode.
Hierfür werden die Abstände zwischen Maxima und Minima gemittelt.
Die Maxima der $T_8$-Kurve aus Abbildung \ref{fig:edelstahldyn} sind schwer zu
bestimmen, deswegen weichen die Werte stärker voneinander ab.

\begin{table}
      \centering
      \caption{Wärmeleitfähigkeiten}
      \label{tab:leitergebnisse}
      \begin{tabular}{c S[table-format=3.0] @{${}\pm{}$} S[table-format=1.0] S[table-format=3.0] S[table-format=2.0]}
            \toprule
            {Material}
            & \multicolumn{2}{c}{$κ_\text{mittel}\:[\si{\watt\per\metre\per\kelvin}]$}
            & {$κ_\text{lit}\:[\si{\watt\per\metre\per\kelvin}]$}
            & {Abweichung\:$[\si{\percent}]$} \\
            \midrule
            Messing   &  76 & 4 &  90 & 18 \\
            Aluminium & 173 & 6 & 220 & 27 \\
            Edelstahl &  31 & 7 &  21 & 47 \\
            \bottomrule
      \end{tabular}
\end{table}
