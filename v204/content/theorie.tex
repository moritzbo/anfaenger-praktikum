\section{Theorie}
\label{sec:theorie}
Existiert in einem Körper ein Temperaturungleichgewicht,
findet ein Wärmetransport in Richtung des kälteren Wärmereservoirs statt.
Dies kann als eine Variation des 2. Hauptsatzes der Thermodynamik aufgefasst
werden.
Hier wird nun angenommen, dass der Wärmetransport allein durch Phononen, welche
als elementare Anregung eines elastischen Feldes verstanden werden können,
und freie Elektronen geschieht.
Die Wärmemenge, welche durch einen Stab der Länge $\symup{L}$ mit der
Querschnittsfläche $\symup{A}$, der Dichte $ρ$ und der spezifischen
Wärmekapazität c, fließt berechnet sich gemäß
\begin{equation}
  \label{eqn:waermemenge}
  \symup{d} Q = - \kappa \symup{A} \frac{\partial \symup{T}}{\partial \symup{x}}
  \symup{d} t \:,
\end{equation}
wobei $κ$ die materialabhängige Wärmeleitfähigkeit ist.
Damit gilt für den Wärmestrom
\begin{equation}
  j_\text{w} = - κ \frac{\partial \symup{T}}{\partial \symup{x}} \:.
  \label{eqn:waermestrom}
\end{equation}
Unter Zuhilfenahme der der Kontinuitätsgleichung lässt sich
\begin{equation}
  \label{eqn:wlg}
  \frac{\partial \symup{T}}{\partial \symup{t}} = \left(\frac{κ}
  {ρ \symup{c}}\right) \frac{\partial{^2} \symup{T}}{\partial \symup{x}^2}
  \:,
\end{equation}
die Wärmeleitungsgleichung\footnote{Die vollständige Wärmeleitungsgleichung
ist eine Differentialgleichung in alle drei Raumdimensionen $(x,y,z)$.},
herleiten \cite{Anleitung}.
Sie beschreibt die Temperaturverteilung innerhalb des Körpers mit der Zeit.
Der Faktor $\left(\sfrac{κ}{ρ \symup{c}}\right) = σ_\text{T}$ wird
Temperaturleitfähigkeit genannt und gibt an, wie schnell sich die Temperatur
innerhalb des Körpers ausgleicht.
\\
Für die dynamische Methode wird eine Temperaturwelle, durch Erwärmen und
Abkühlen des Körpers, erzeugt. Diese hat die Form
\begin{equation}
  \label{eqn:wärmewelle}
  T (x,t) = T_\text{max} \: \symup{exp} \left( -\sqrt{\frac{ω ρ \symup{c}}
  {2 κ}} \: \symup{x} \right) \cos\left( ω t -
  \sqrt{\frac{ω ρ \symup{c}}{2 κ}} \symup{x} \right) \:.
\end{equation}
Für die Phasengeschwindigkeit der Welle muss die Dispersionsrelation
\begin{equation}
  \label{eqn:phvelocity}
  \symup{v}_\text{Phase} = \frac{ω}{k} = ω \:/\:
  \sqrt{\frac{ω ρ \symup{c}}{2 κ}} = \sqrt{\frac{2 κ ω}
  {ρ \symup{c}}}
\end{equation}
gelten.
\\
Der Dämpfungsfaktor wird aus dem Amplitudenverhältnis $\symup{A}_\text{nah}$
und $\symup{A}_\text{fern}$ an den beiden Messstellen $\symup{x}_\text{nah}$
und $\symup{x}_\text{fern}$ bestimmt.
Die Wärmeleitfähigkeit $κ$ lautet mit
\begin{align}
      ω &= \frac{2 \symup{π}}{\symup{T}}
      \shortintertext{und}
      Φ &= \frac{2 \symup{π} \increment t}{\symup{T}}
      \label{eqn:phase}\\
      κ &= \frac{ρ \symup{c} \left(\increment\symup{x} \right)^2}
      {2 \increment t \: ln \left(\symup{A}_\text{nah}/\symup{A}_\text{fern}\right)}\:.
      \label{eqn:kappa}
\end{align}
Der Abstand der beiden Messstellen zueinander ist dabei $\increment \symup{x}$
und $\increment t$ ist die Phasenverschiebung der Temperaturwelle an den
beiden Messstellen.
