\section{Versuchsaufbau}
In den folgenden Experimenten wird die magnetische Flussdichte längs der Achsen verschiedener Spulentypen gemessen und die Ergebnisse werden im Anschluss dargestellt. \\
Generell werden die einzelnen Experimente im spanunngslosen Zustand aufgebaut und vor dem Einschalten sind Strom und Spannung auf Null geregelt.
Es darf nur die maximal zulässige Stromstärke benutzt werden. Zum Messen der Feldstärke wird die transversale oder longitudinale Hall-Sonde benutzt.
Hall-Sonden funktionieren nach dem Messprinzip des Hall-Effekts. Ein Leiterplättchen an der Spitze der Sonde ist an Steuerstrom angelegt.
Ein senkrecht zum Steuerstrom gerichtetes Magnetfeld erfährt, durch die Lorentzkraft, die auf die Ladung wirkt, einen Verschiebungsstrom.
Senkrecht zum Strom und zum Magnetfeld wird eine Spannung aufgebaut(Hall-Spannung). Konstanter Steuerstrom ist eine Voraussetzung für die Hall-Spannung als Maß für die Stärke des Magnetfeldes.

\section{Durchführung}
\subsection{Magnetfeldmessung von langestreckten Spulen}
Die Langspulen werden an das Netzgerät angeschlossen.
Der Strom wird auf 1 Ampere geregelt. Zum Messen des Magnetfeldes wird eine longitudinale Sonde verwendet.
Die Justierung der Sonde erfolgt so, dass sie das Magnetfeld auf der Achse der Spule misst.
Es werden Messwerte innerhalb und außerhalb der Spule abgenommen.
Es werden Messreihen mit einer $\SI{5.5}{\centi\metre}$ langen und einer $\SI{16}{\centi\metre}$ langen Spule vorgenommen.

\subsection{Messungen am Helmholtzspulenpaar}

Zunächst wird das Netzgerät an das in Reihe geschaltete Spulenpaar angeschlossen. Der Strom wird auf 3 Ampere geregelt.
Die Messreihen umfassen 3 verschiedene Spulenabstände.
Sie variieren zwischen 7, 8 und $\SI{9}{\centi\metre}$, bei einem mittleren Spulendurchmesser von d = $\SI{12.5}{\centi\metre}$, einer Windungszahl von n = 100 und einer Spulenbreite von b = $\SI{3.3}{\centi\metre}$.
Es wird mit einer transversalen Hall-Sonde gemessen, mit der innerhalb und außerhalb des Spulenpaares das Magnetfeld gemessen wird.

\subsection{Messung eines Magnetfeldes an der Ringspule}

Es wird ein Netzgerät an einer Ringspule mit Luftspalt angeschlossen.
Die transversale Hall-Sonde wird in den Luftspalt eingeführt und das Magnetfeld wird als Funktion des Spulenstroms gemessen.
Die Ringspule mit Luftspalt verfügt über die Windungszahl n = 595 und eine Luftspaltbreite von b = $\SI{0.3}{\centi\metre}$.
Dabei wurde der Strom von 0 auf $\SI{10}{\ampere}$, in $\SI{1}{\ampere}$-Schritten hochgeregelt und anschließend wieder heruntergereget.
Dann wird die Stromrichtung umgedreht und analog verfahren. Anschließend wird
die Stromrichtung wieder gedreht und erneut in $\SI{1}{\ampere}$-Schritten gemessen, bis $\SI{10}{\ampere}$.
