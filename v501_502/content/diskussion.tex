\section{Diskussion}
\label{sec:Diskussion}
Die Auswertung zu Versuch 501 liefert bezüglich der
Empfindlichkeitsmessung eine prozentuale Abweichung zum Literaturwert von
$\SI{2.9}{\percent}$, der Literaturwert liegt in der 1-$σ$-Umgebung unseres Wertes.
Die Abweichung kann an der Bündelung des Strahles liegen,
die nicht über die Höhe des Schirms gleich blieb.
Eine weitere Fehlerquelle sind die Ablenkplatten im Kathodenstrahlrohr
welche nicht vollständig parallel sind, sondern an den Enden etwas auseinandergehen.
Dem zufolge ist das elektrische Feld am Ende der Platten inhomogen, was in einer
Störung der Elektronenbahn resultiert. Zudem traf der Elektronenstrahl ohne
anliegende Felder nicht mittig auf dem Schirm auf.
\\
Mithilfe des Kathodenstrahl-Oszillographen wurde die Sinusfrequenz zu
$\SI{79.84(002)}{\hertz}$ bestimmt. Der angegebene Frequenzbereich vom Generator liegt
zwischen $\SI{80}{\hertz}$ und $\SI{90}{\hertz}$, unsere Berechnung
passt somit zur Vorgabe. Eine Fehlerquelle, welche die Auswertung erschwert, ist
die Tatsache, dass keine sauberen Frequenzverhältnisse einzustellen sind,
da nur ein Bruchteil der Schwingungsbildes auf dem Schirm zu erkennen war.
Zudem konnten nur drei Werte genommen werden.
\\
Die in Versuch 502 bestimmte spezifische Elektronenladung weicht um
$\SI{107}{\percent}$ ab.
Der Mittelwert ist
\begin{align}
      &\SI{-3.65(4)e11}{\coulomb\per\kilo\per\gram}\;,
      \intertext{der Literaturwert aus \cite{elektron} ist}
      &\SI{-1.758820024e11}{\coulomb\per\kilo\per\gram}\;.
\end{align}
Die Abweichung des Wertes stammt aus einem unbekannten systematischen Fehler,
da die beiden berechneten Werte nicht stark voneinander abweichen.
\\
Die Horizontalkomponente des Erdmagnetfeldes liefert $\SI{59.49}{\micro\tesla}$.
Die Abweichung zum Literaturwert von $\SI{48.99}{\micro\tesla}$\cite{bfield} beträgt
$\SI{21}{\percent}$, nach Formel \eqref{eqn:prozfehler}. Die Messung ist erstaunlich genau,
da der Kompass zur
Messung des Inklinationswinkel sehr schwergängig funktioniert hat und auch weitere
Bemühungen einen vernünftigen Winkel zu erlangen nicht genau waren.
