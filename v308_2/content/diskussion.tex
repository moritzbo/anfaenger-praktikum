\section{Diskussion}
\label{sec:Diskussion}
Zusammenfassend lässt sich sagen, dass die Messwerte in guter Näherung die
Theoriewerte widerspiegeln, dennoch gibt es Abweichungen.
Bei der langen Spule ist das Magnetfeld, wie zu erwarten, symmetrisch um den
Mittelpunkt der Spule, welcher in Abbildung \ref{fig:spuleL} bei
$x = \SI{13}{\centi\meter}$ liegt. Anders als bei der Theoriekurve wird das
Magnetfeld an den Rändern schneller schwach, was möglicherweise an den Kabeln
liegt, da diese selbst laut dem Gesetz von Biot-Savart ein Magnetfeld erzeugen
und ggf.\footnote{gegebenenfalls} das Magnetfeld der Spule schwächt. Außerdem
konnte eine Hitzeentwicklung der Spule wahrgenommen werden. Die
Temperaturerhöhung erhöht den Widerstand und schwächt auch die magnetische
Feldstärke.

Bei der kurzen Spule ist das Magnetfeld innerhalb nicht konstant, genauso wie
die Theoriewerte dies auch aufzeigen. Viel komischer ist, dass die kurze mit
$l = \SI{5.5}{\centi\meter}$ nach $x = \SI{5}{\centi\meter}$ anfängt, wo die
Kurve der Theoriewerte in etwa ihr Maximum hat. (vielleicht ein python fehler)
Da das Maximum der Messwerte um $\SI{0.2}{\milli\tesla}$ größer ist,
liegt die Vermutung nahe, dass entweder die Hallsonde fehlerbehaftet ist oder
das Magnetfeld nicht exakt im Mittelpunkt der Spule gemessen wurde. Auch ein
ungenaues Ablesen der Messwerte, verrutschen der Apparatur oder Störeinflüsse
durch die Experimente der anderen Gruppe können zu Unsicherheiten führen.
Dies würde vielleicht den Offset der Maxima der Kurven erklären.

Bei den Berechnungen der magnetischen Flussdichte für das Helmholtzspulenpaar
liegen die theoretischen Werte über den gemessenen Werten. Dies liegt womöglich
wieder an den zusätzlichen, durch die Kabel verursachten, Magnetfelder,
Ablesefehler oder daran, dass die Hallsonde nicht exakt im $\SI{90}{\degree}$
Winkel zu den Magnetfeldlinien gehalten wurde, wodurch das Kreuzprodukt in
der Lorentzkraft nicht exakt maximal wurde. Dies gilt für beide Messungen.

Bei der Aufnahme der Hysteresekurve wurden nur alle $\SI{1}{\ampere}$
Messwerte aufgenommen. Dies führt dazu, dass die Kurve nicht besonders genau ist.
Außerdem wurde die benutzte Toroidspule schon häufiger benutzt und konnte
nicht exakt entmagnetisiert werden. Dies führt auf eine andere Hysteresekurve
als die aufgenommene. Zusätzlich stecken in der Hallsonde und dem Amperemeter
eine gewisse Unsicherheit, sodass die Werte etwas abweichen können. Wieder können
auch von außen wirkende Störfelder einen Einfluss auf das Aussehen der Kurve
haben.
