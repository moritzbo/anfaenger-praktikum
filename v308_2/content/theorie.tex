\section{Theorie}
\label{sec:theorie}
\subsection{Das Magnetfeld}
Wenn elektrische Ladungen bewegt werden, erzeugen sie ein magnetisches Feld
$\symbf{B}$.
Sobald ein Leiter mit Strom durchflossen wird, erzeugt dieser ein magnetisches
Feld. Die magnetischen Feldlinien sind immer geschlossen. Das Feld ist also ein
Rotationsfeld. Dabei bilden die Feldlinien konzentrische Kreise um den Leiter.
Die Rotationsrichtung ist durch die \enquote{rechte Hand Regel} gegeben, wobei
die Feldlinien im rechten Winkel zur Stromrichtung eine Rechtsschraube
beschreiben.
Die Magnetfeldstärke $\symbf{B}$ eines stromdurchflossenen Leiters lässt
sich mit dem Gesetz von Biot-Savart bestimmen.
\begin{equation}
  \symup{d}\symbf{B} =
  \frac{\mu_0 \symbf{I}}{4 \symup{\pi}}
  \frac{\symup{d}\vec{s} \times \vec{r}}{r^3}
\label{eqn:biotleiter}
\end{equation}
Durch Integration folgt daraus
\begin{equation}
  \symbf{B}(\symbf{r}) = \frac{\mu_0}{4 \symup{\pi}}
    \int_C \frac{I \symup{d} \symbf{s} \times \symbf{r}}{r^3}\,.
  \label{eqn:biotsavart}
\end{equation}
\\
Um ein homogenes Magnetfeld zu erhalten, schaltet man zwei Kreispulen so in
Reihe, dass ihre Symmetrieachsen aufeinanderfallen. Wählt man den Abstand $x$
der beiden Spulen zueinander genau gleich dem Spulenradius $\symup{R}$, ist das Feld
innerhalb der beiden Spulen weitesgehend homogen bis auf Randeffekte.
Dieser Aufbau wird Helmholtzspulenpaar genannt.
Das Feld im inneren erhällt man durch Superposition der Einzelfelder der Spulen.
\begin{align}
  B(0) &= B_1(x) + B_1(-x) \\
       &= \frac{\mu_0 \symbf{I} R^2}{\left(R^2 +x^2\right)^{3/2}}
\end{align}
\\
Der Betrag der magnetischen Flussdichte einer langen Spule auf der
Symmetrieachse in x-Richtung lässt sich mit Hilfe von Gleichung
\eqref{eqn:biotsavart} und
\begin{equation}
  \symup{B} = \mu_\text{r} \mu_0 \symbf{I} \frac{n}{l}
\end{equation}
zu
\begin{equation}
  B_x(x) = \frac{\mu_0 n I}{2} \left( \frac{x-x_1}{\sqrt{(x-x_1)^2+R^2}}
  + \frac{x-x_2}{\sqrt{(x-x_2)^2+R^2}} \right)
  \label{eqn:spuleGenau}
\end{equation}
berechnen\footnote{Die genaue Rechnung kann unter \cite{solenoid} im 
Literaturverweis nachvollzogen werden.}.
Der magnetische Fluss der idealen Helmholtzspule lässt sich gemäß
\begin{equation}
  B_x(x) = \frac{\mu_0 \symbf{I}}{2 r} \left( \frac{1}
  {(\gamma^2+\gamma+5/4)^{3/2}} + \frac{1}{(\gamma^2-\gamma+5/4)^{3/2}}
  \right)
  \label{eqn:helmholz}
\end{equation}
berechnen. Dabei ist $\gamma$ das Verhältnis des betrachteten Ortes mit dem
Spulenradius, also $\gamma = x / r$, $\mu_0$ ist die magnetische Feldkonstante
mit $\mu_0 = 4 \symup{\pi}\cdot\num{e-7}$ und $\symbf{I}$ ist der Strom,
welcher die Spulen durchfließt\cite{helm}. $\mu_\text{r}$ bezeichnet die
Fähigkeit eines Stoffes ein Magnetfeld zu verändern, wenn es innerhalb einer
Spule platziert wird. Die Materialien können das Feld abschwächen
($\mu_\text{r,Zink}=0.99$) oder verstärken ($\mu_\text{r, Eisen}=8000$)
\cite{muR}.

\subsection{Die Hallsonde}
Die Hallsonde ist ein Messinstrument um die magnetische Flussdichte von
verschiedenen Leitern zu bestimmen.
An dem Sensor der Sonde befindet sich ein Leiterplättchen an welches ein
Steuerstrom angelegt wird. Hält man die Hallsonde nun senkrecht zu den
Feldlinien in das Feld wirkt die Lorentz-Kraft
\begin{equation}
  \symup{F}_\text{L} = \symup{q} \symbf{v} \times \symbf{B}
\end{equation}
auf die Ladungen. Dies resultiert in einem Verschiebungsstrom. Dieser bezweckt
die sogenannte Hallspannung. Sie ist ein Maß für die Stärke eines Magnetfeldes.

\subsection{Hysterese}
Ferromagnetische Stoffe wie zum Beispiel Eisen bestehen im wesentlichen aus
winzigen, sogenannten Weiß'schen Bezirken, welche ohne ein äußeres Feld
statistisch gerichtet sind. Sobald ein Feld von außen angelegt wird,
\enquote{klappen} diese nacheinander sprungartig um bis alle in die selbe
Richtung wie das angelegte Feld zeigen. Wenn das Feld nun abgeschaltet wird, \enquote{klappen} die
Bezirke wieder zurück, es bleibt aber eine Restmagnetisierung im Material.
Das heißt die magnetischen Eigenschaften eines ferromagnetischen Stoffes hängen
von seiner Vorgeschichte ab.
Dieses Verhalten sammt Vorgeschichte werden durch die Hysteresekurve
dargestellt.
Die Hysteresekurve eines Materials zeigt an, wie es sich in einem äußeren
Magnetfeld verhält.
Dabei sind drei wichtige Merkmale zu erkennen:
Die Sättigungsmagnetisierung, welche den extremalsten Wert der Feldstärke
darstellt, die der Ferromagnet bei einem gegebenen Wert der Stromstärke
erreichen kann.
Dann die Koerzitivfeldstärke bzw. Koerzitivkraft. Dies ist die Feldstärke die
benötigt wird um einem zuvor bis zur Sättigungsmagnetisierung aufgeladenen
ferromagnetischen Stoff, vollständig zu entmagnetisieren.
Zuletzt die Remanenz oder Restmagnetisierung, welche zurückbleibt wenn man das
äußere Feld abschaltet\cite{Anleitung}.
