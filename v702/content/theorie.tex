\section{Theorie}
\label{sec:theorie}
\subsection{Aktivierung mit Neutronen}
Die Aktivierung des zu vermessenden Kerns muss direkt vor der Messung geschehen,
da die Halbwertszeit $T \leq \SI{1}{\hour}$ ist. Die Halbwertszeit $T$ ist die
Zeit, in der die Hälfte der Kerne der Anfangsmenge zerfallen ist.
Zur Anregung der Kerne werden Neutronen statt Protonen verwendet,
da diese nicht vom Kern abgestoßen werden.
Die Neutronen stammen aus dem Beschuss von $\ce{^{9}_{4}Be}$ mit $α$-Teilchen,
diese stammen aus dem natürlichen Zerfall von $\ce{^{226}Ra}$.
Die Erzeugung der Neutronen läuft nach
\begin{equation}
    \ce{^{9}_{4}Be + ^{4}_{2}\symup{α} -> ^{12}_{6}C + ^{1}_{0}n}
\end{equation}
ab, \cite{Anleitung}.
Die Neutronen werden zunächst abgebremst, da somit der
Wirkungsquerschnitt $σ$ vergrößert wird.
Dieser sagt aus wie wahrscheinlich ein Neutron von einem Kern eingefangen wird.
Durch eine kleinere Geschwindigkeit
ist die Energie ebenfalls kleiner, durch elastische Stöße mit möglichst leichten
Atomen kann dies realisiert werden. In diesem Versuch wird Paraffin verwendet.
Durch die niedrige Energie verweilen die Neutronen, bildlich gesehen,
länger in der Nähe des Kerns, der Wirkungsquerschnitt wird dadurch größer.

Ein angeregter Kern wird Zwischenkern, oder auch Compoundkern genannt,
er befindet sich etwa $\SI{e-16}{\second}$ in diesem Zustand.
Mit Emission eines $γ$-Quants kehrt der Kern aus dem angeregten Zustand in den
Grundzustand.
\begin{equation}
    \ce{^{m}_{z}A + ^{1}_{0}n -> ^{m+1}_{z}A^* -> ^{m+1}_{z}A + \symup{γ}}
\end{equation}
Die Anregung erfolgt durch Einfangen eines Neutrons, dessen
Energie auf alle Teilchen im Kern gleichmäßig verteilt wird, sowie in die
Bindungsenergien übergeht.

Mit einem $\ce{\symup{β}-}$-Zerfall zerfällt der Kern zu einem stabilen Kern,
unter Emission eines Elektrons und eines Anti-Elektron-Neutrino,
\begin{equation}
    \ce{^{m+1}_{z}A -> ^{m+1}_{z+1}C + \symup{β}-} + E_\text{kin} + \symup{\bar{ν}}_\text{e}
\end{equation}
Der Zerfall findet statt, wenn die Neutronenanzahl zu hoch für einen stabilen
Kern ist. Bei den verwendeten Isotopen ist das bei einem Neutron mehr der Fall.
\newpage
\subsection{Verwendete Messmethode}
\label{sec:messmethode}
Da die Messung der Zerfallskurve
\begin{align}
    N(t) &= N_0 \symup{e}^{-λt}
    \label{eqn:zerfallskurve}
    \intertext{an sich nur mit großen Fehlern messbar ist, wird in diesem Versuch}
    N_{\increment t}(t) &= N(t) - N(t+\increment t)
    \intertext{gemessen. Anders geschrieben}
    N_{\increment t}(t) &= N_0 \left(1-\symup{e}^{-λ\,\increment t}\right) \symup{e}^{-λt}\:.
    \intertext{Konstant ist der Ausdruck}
    & N_0 \left(1-\symup{e}^{-λ\,\increment t}\right)\:,
    \intertext{sodass über die Gesamtformel die Zerfallskonstante $λ$ bestimmt werden kann.
    $λ$ kann zudem über die Halbwertszeit $T_\text{h}$ bestimmt werden}
    N(T_\text{h}) &= \frac{N_0}{2} = N_0 \symup{e}^{-λT_\text{h}} \\
    T_\text{h} &= \frac{\ln(2)}{λ}\:. \label{eqn:halbwert}
\end{align}


\subsection{Zerfälle}
Das verwendete Indium Präparat zerfällt nach
\begin{align}
    \ce{^{115}_{49}In + ^{1}_{0}n &-> ^{116}_{49}In -> ^{116}_{50}Sn + \symup{β}- + \bar{\symup{ν}}_e}\:.
\end{align}
Bei Rhodium kann ein instabiles höher energetisches Isomer entstehen, dieses
zerfällt mit einem $γ$-Quant in das andere Zerfallsprodukt.
Diese beiden Zerfälle können getrennt werden, da der Übergang des 104i zum 104
eine kürzere Halbwertszeit hat, als der Übergang von Rhodium zu Palladium.
\begin{equation}
    \ce{^{103}_{45}Rh + ^{1}_{0}n}
    \begin{cases}
        \xrightarrow{\;\;\SI{10}{\percent}\;\;} \ce{^{104i}_{45}Rh -> ^{104}_{45}Rh + \symup{γ} -> ^{104}_{46}Pd + \symup{β}- + \bar{\symup{ν}}_e} \\
        \xrightarrow{\;\;\SI{90}{\percent}\;\;} \ce{^{104}_{45}Rh -> ^{104}_{46}Pd + \symup{β}- + \bar{\symup{ν}}_e}
    \end{cases}
\end{equation}
