\section{Diskussion}
\label{sec:Diskussion}
Über die Genauigkeit der Messung der Untergrundstrahlung, welche durch zum Beispiel
$\ce{^{40}K}$ auftritt, kann nicht viel gesagt werden.
Das ist die erste Fehlerquelle. Außerdem musste bei
jeder Messung davon ausgegangen werden, dass die Totzeit des Geiger-Müller-Zählrohrs
hinreichend klein ist, sodass dieses einen verschwindend kleinen Fehler einbringt.

Die prozentualen Abweichungen der Halbwertszeiten sind relativ groß,
die $σ$-Umgebungen der Werte passen noch weniger mit den Literatrwerten überein.
In Tabelle \ref{tab:params} ist zu erkennen, dass die Übereinstimmung von
$σ$-Umgebung und Literaturwert bei $\ce{^{104}Rh}$ mit einer 4-$σ$-Umgebung am
kleinsten ist. Gerade die Rhodium-Mesung ist durch die beiden Zerfallsreihen
in der Auswertung schwierig.

Aus den Abbildungen kann entnommen werden, dass die Zählungen bei Rhodium,
für den hinteren Bereich, sehr mit statistischen Fehlern behaftet sind
und deshalb der Fit nicht optimal ist.
Dennoch liegen die Fitkurven in den Abbildungen gut zu den Messwerten.

\begin{table}
  \centering
  \caption{Halbwertszeiten und prozentuale Abweichung. \cite{litInd} \cite{litRh}}
  \label{tab:params}
  \begin{tabular}{c| c c S[table-format=2.2]}
    \toprule
    {Isotop}
    & {$T_\text{h}$}
    & {$T_\text{h,lit}$}
    & {$\increment T_\text{h}\:/\:\si{\percent}$} \\
    \midrule
    $\ce{^{115}In}$  & $\SI{60.7(3)}{\minute}$  & $\SI{54.28}{\minute}$ & 11.83 \\
    $\ce{^{104}Rh}$  & $\SI{6.0(5)}{\minute}$   & $\SI{4.3}{\minute}$   & 39.53 \\
    $\ce{^{104i}Rh}$ & $\SI{32.6(17)}{\second}$ & $\SI{42.3}{\second}$  & 22.93 \\
    \bottomrule
  \end{tabular}
\end{table}
