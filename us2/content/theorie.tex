\section{Theorie}
\label{sec:theorie}
Ultraschall liegt in der Frequenz über dem hörbaren Schall,
folglich beginnt der Ultraschallbereich bei $ν = \SI{20}{\kilo\hertz}$.
Er endet beim Übergang zum Hyperschall bei $ν = \SI{1}{\giga\hertz}$.

\subsection{Ultraschall}
Die Ultraschallwellen breiten sich, wie Schallwellen, in Luft und Gasen
longitudinal aus.
Die Amplituden sind Druckschwankungen, sodass die Ultraschalltechnik in der
Werkstoffprüfung als Methode der zerstörungsfreien Prüfung verwendet werden kann.

Ultraschallwellen können mit Piezokristallen erzeugt werden.
Diese werden in einem elektrischen Feld zu Schwingungen angeregt,
sodass Ultraschallwellen abgestrahlt werden.
Dieser piezo-elektrische Effekt kann auch zum detektieren von Ultraschallwellen
verwendet werden.

\subsection{Beschreibung durch Wellenfunktionen}
Die Wellenfunktion, nach \cite{Anleitung},
\begin{align}
    p(x,t) &= p_0 + v_0\:Z\:\cos(ω\:t-k\:x)
    \intertext{beschreibt die Ausbreitung der longitudinalen Schallwelle.}
    Z&=c\cdotρ
    \intertext{ist die akustische Impedanz des jeweiligen Materials.
    Dabei ist $c$ die Schallgeschwindigkeit im Medium:}
    c_\text{Fl} &= \sqrt{\frac{1}{κ\cdotρ}}\:,
    \intertext{mit der Kompressibilität $κ$ des Stoffes und der Dichte $ρ$.
    In Festkörpern ist die Schallausbreitung nicht rein longitudinal, sondern
    durch Schubspannungen auch transversal. Für die Schallgeschwindigkeit wird
    daher nicht die Kompressibilität, sondern das Elastizitätsmodul verwendet}
    c_\text{Fest} &= \sqrt{\frac{E}{ρ}}\:.
    \intertext{Durch die Streuungen und Absorptionen im Festkörper nimmt die
    Intensität der Welle exponentiell mit}
    I(x) &= I_0 \cdot \symup{e}^{-α\:x}
    \label{eqn:alpha}
\end{align}
ab.

\subsection{Reflexion und Transmission}
Die durch eine Grenzschicht transmitierte Welle hat den Anteil
\begin{align}
    T &= 1 - R
    \intertext{an der einfallenden Welle. $R$ ist der Reflexionskoeffizient}
    R &= \left(\frac{Z_1-Z_2}{Z_1+Z_2}\right)^{\!\!2}\:.
    \intertext{Mit den jewiligen akustischen Impedanzen}
    Z_i &= c_i \cdot ρ_i\:.
\end{align}

\subsection{Messverfahren}
Beim messen mit Ultraschall gibt es verschieden Methoden wie die Messwerte genommen
werden, sowie welche Messwerte beim quasi gleichen Messaufbau genommen werden.

\subsubsection{Durchschallungsverfahren}
Beim Durchschallungsverfahren wird der Ultraschall in eine Probe geleitet und
auf der gegenüberliegenden Seite vom Sender aufgefangen. Wenn die Probe inhomogen
ist, kann dies durch die abgeschwächte Intensität erkannt werden. Weitere
Ausssagen, wie die Position der Inhomogenität sind nicht möglich.

\subsubsection{Impuls-Echo-Verfahren}
Wie der Name des Verfahrens sagt, wird eine Ultraschallwelle als Impuls in
das Probestück geschickt, durch die Reflexion an den Grenzflächen von Inhomogenitäten
entstehen zurücklaufende Wellen. Diese werden von der Ultraschallsonde detektiert.

Durch die Laufzeit $t$ kann bei bekannter Schallgeschwindigkeit im Material die
Strecke bis zur Verunreinigung nach
\begin{align}
    s &= \frac{1}{2}\cdot c \cdot t
    \intertext{berechnet werden. Ist die Länge der Probe bekannt,
    kann alternativ auch die Schallgeschwindigkeit in dem Medium bestimmt werden:}
    c &= 2\cdot\frac{s}{t}\:.
    \label{eqn:velocity}
\end{align}

\subsection{Scanmethoden}
Für das Impuls-Echo-Verfahren gibt es verschiedene Auswertungsmethoden.

A-Scan steht für Amplitudenscan, hierbei werden die Amplituden betrachtet,
um auch die Abschwächung bestimmen zu können.

Bei einem B-Scan wird ein zweidimensionales Graustufenbild aufgenommen.
Der Grauton steht dann für die Intensität der zurücklaufenden Welle.

Beim TM-Scan wird ein Ultraschallsignal mit hoher Wiederholungsfrequenz verwendet.
Durch die schnelle Abfolge an Signalen kann ein Videoähnliches Signal
aufgenommen werden.

\subsection{Herzvolumen}
Um das Herzvolumen des Modells zu bestimmen, wird das Herz als Zylinder angenommen.
Das Volumen ist demnach
\begin{equation}
  V = \symup{\pi} \cdot r^2 \cdot h\:.
  \label{eqn:herzvol}
\end{equation}
Durch das Pumpen wird das Modell aufgeblasen. Mit dem A-Scan kann die
Amplitude bestimmt werden und damit das Volumen des Herzens.
