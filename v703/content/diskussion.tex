\section{Diskussion}
\label{sec:Diskussion}
Die Auswertung der Charakeristik des Zählrohres ergab ein Plateau
im Bereich von $390-\SI{610}{\volt}$, welches keine
einheitliche Steigung aufweist. Demnach ist der Fehler in der linearen
Ausgleichsrechnung relativ groß. Der Grund der Fluktuation der Zählungen ist
statistischer Natur und rührt daher, dass auch Nachentladungen auftreten, welche
zu Unsicherheiten führen. Insgesamt war es zum Zeitpunkt der Durchührung sehr warm
im Raum, was zum Ausfall eines anderen Impulszählers führte, die Auswirkungen auf
unser Gerät sind unbekannt.
\\
Die Messung der Nachentladungen ergab, dass mit größer werdender Spannung, auch
die Dauer der Erholungszeit größer wird. Dieses Ergebnis ist nicht logisch,
da die Dauer bis zur Nachentladung davon abhängt, wie schnell die
Ionen zur Kathode gelangen und die dann herausgelösten Elektronen die Anode
erreichen. Da  mit größer werdender Spannung die Teilchenbewegungen
schneller werden, ist das aufgenommene Ergebnis ein Widerspruch zur Theorie.
Da die Nachentladungen über das Oszilloskop aufgenommen wurden, liegt es
vermutlich daran, dass das Oszilloskopbild sehr schwer auszuwerten war.
\\
Bei der Totzeitmessung liegen die Ergebnisse der Oszilloskopmessung und der
Zwei-Quellen-Methode $\SI{40}{\micro\second}$, mit maximalen Fehlern, auseinander.
Das ist bei einem Wert von
$\overline{T}_\text{t,O} = \SI{184(9)}{\micro\second}$,
für die Messung mit dem Oszilloskop,
eine große Diskrepanz.
Diese ist anscheinend weniger Vorteilhaft als die Zwei-Quellen-Methode,
basierend auf dem Korrekturkommentar und dem Literaturwert \cite{totzeit},
$T_\text{t,l}=\SI{100}{\micro\second}$.
