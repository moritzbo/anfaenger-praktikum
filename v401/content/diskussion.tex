\section{Diskussion}
\label{sec:Diskussion}
Die Messung der Wellenlänge liefert ein Ergebnis, welches nur noch einen
statistischen Fehler aufweist, da der Literaturwert in der 1-$σ$-Umgebung
unseres Messwertes liegt. Die systematischen Fehler, die durch den Aufbau
bedingt sind, wie die Zählung der Maxima, konnten durch Anpassung der
Wegunterschiede gering gehalten werden. Ebenso die Erschütterungen durch
'rumlaufen' wurden gering gehalten.

\begin{table}
      \centering
      \caption{Messwerte und Literaturwerte.\cite{brechlit}}
      \label{tab:werteml}
      \begin{tabular}{c
            S[table-format=3.0] @{${}\pm{}$} S[table-format=2.0]
            S[table-format=1.6] @{${}\pm{}$} S[table-format=1.6]
            }
            \toprule
            {} & \multicolumn{2}{c}{$λ\:/\:\si{\nano\metre}$}
            & \multicolumn{2}{c}{$n$} \\
            \midrule
            {Messwert} & 638 & 22 & 1.000294 & 0.000003 \\
            {Literaturwert} & 650 & 0 & 1.000272  & 0\\
            \bottomrule
      \end{tabular}
\end{table}

Der Literaturwert der Brechungsindexmessung liegt in der 7-$σ$-Umgebung
unseres Brechungsindexes. Die großen Fehlerquellen liegen hier nicht bei dem
Zählwerk und der Photodiode, da durch das kontrollierte Zurückfließen der Luft,
die Maxima langsam genug für das Zählwerk waren.
Die Fehlerquelle hier ist die Hand-Vakuum-Pumpe, da die Drücke nicht genau
abgelesen werden können. Die Nachkommastellen konnten nur auf eine Stelle genau
abgelesen werden, denn die Skala war nicht genauer. Für eine bessere Bestimmung
des Brechungsindexes fehlen hier die Stellen.
\\~\\
Die Messung der Wellenlänge liefert ein gutes Ergebnis, für den Brechungsindex
sollte die Messung mit einer besser ablesbaren Vakuumpumpe wiederholt werden.
