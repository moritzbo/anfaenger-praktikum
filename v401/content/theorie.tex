\section{Theorie, nach \cite{Anleitung}}
\label{sec:theorie}
\subsection{Allgemeines}
Das Michelson-Interferometer beruht auf dem Prinzip der Interferenz
und dient dazu, Wellenlängen und Brechungsindizes
zu bestimmen.
Dabei wird jeweils eine Weglängenänderungen vorgenommen, wodurch an der
Photodiode Interferenzmaxima vorbeilaufen und detektiert werden.
Dies passiert bei der Wellenlängenmessung durch Spiegeleinstellungen und bei
der Messung des Brechungsindexes durch ein Gas.
Voraussetzung für Interferenz ist die Kohärenz des Lichtes. Das bedeutet, dass
die Phasenkonstanten $δ_\text{i}$ keine statistischen Funktionen sein dürfen,
sondern zeitlich konstant sein müssen. Hier wird ein Diodenlaser\footnote{light
amplification by stimulated emission of radiation} verwendet um kohärentes Licht
zu erzeugen.

\subsection{Interferenz}
Interferenzerscheinungen treten dann auf, wenn zwei Wellen zueinander
phasenverschoben sind.
Da wegen der hohen Frequenz des Lichtes die Feldstärke nicht
direkt gemessen werden kann, wird stattdessen die Intensität $\symbf{I},$
die proportional zum Amplitudenquadrat ist, gemessen.
\\
Es tritt konstruktive Interferenz auf, wenn der Wegunterschied der Teilstrahlen
$\symup{Δ}s$ ein geradzahliges Vielfaches von $λ\:/\:2$ ist, und destruktive
Interferenz, wenn der Wegunterschied ein ungeradzahliges Vielfaches von
$λ\:/\:2$ ist.
\begin{align}
  \symup{Δ}\symbf{I}_\text{max}\:\text{für}\;\symup{Δ}s &= 2n\frac{λ}{2} = n \cdot λ\\
  \symup{Δ}\symbf{I}_\text{min}\:\text{für}\;\symup{Δ}s &=
  \left( 2n + 1 \right)\: \frac{λ}{2}
\end{align}
Wichtig ist, dass der Wegunterschied $\symup{Δ}s$ nicht größer als die Kohärenzlänge
$l$ ist, weil sonst keine Interferenz mehr messbar ist.
Sie berechnet sich gemäß
\begin{equation}
  l = N \cdot λ \:.
\end{equation}

\subsection{Bestimmung der Wellenlänge}
Um aus der Versuchsanordnung die Wellenlänge der Lichtquelle zu bestimmen wird
angenommen, dass es sich um ebene Wellen der Form
\begin{align}
  \vec{\symup{E}}_1 &= \vec{\symup{E}}_0 \exp \left( \symup{ikx} \right) \\
  \vec{\symup{E}}_2 &= \vec{\symup{E}}_0 \exp \left( \symup{ikx}
                      \: + \: 2\symup{d} \: + \: \symup{π} \right)
\end{align}
handelt.
Auf einem Schirm werden Minima und Maxima als Streifenmuster erkennbar.
Mit einem Zählwerk wird die Anzahl $Z$ der an der Photodiode pro $\symup{Δd}$
vorbeiziehenden Maxima bestimmt.
Die Wellenlänge bestimmt sich dann gemäß
\begin{equation}
  \symup{λ} = \frac{2 \symup{Δd}}{\symup{z}}\:.
  \label{eqn:lambda}
\end{equation}

\subsection{Bestimmung des Brechungsindexes}
Die gleiche Apparatur wie zur Messung der Wellenlänge kann auch verwendet werden
um den Brechungsindex eines Gases zu ermitteln. Dafür wird die evakuierte
Messzelle der Breite $b$ mit dem jeweiligen Gas befüllt, was durch die
Veränderung der optischen Weglänge zu Interferenz führt. Beim Einfüllen des
Gases erhöht sich der Druck im Inneren um $\symup{Δ}p$, welches den
Brechungsindex um $\symup{Δ}n$ erhöht. Daraus folgt dann
\begin{equation}
  b \cdot \symup{Δ}n = z \frac{λ}{2}\:.
  \label{eqn:deltan}
\end{equation}
Da es sich nur um kleine Drücke handelt, gilt die ideale Gasgleichung
\begin{equation}
  \symup{p\,V} = \symup{R\,T}\:.
  \label{eqn:gasgleichung}
\end{equation}
Wird der Druck von $p$ auf $p'$ verändert, ergibt sich folgender
Zusammenhang für $\symup{Δn}$:
\begin{equation}
  \symup{Δn}(p,p') = \frac{f}{2}\,\symup{N}_\text{L}\,
  \frac{\symup{T}_0}{\symup{p}_0}\,\frac{1}{T}\,(p-p')\;.
\end{equation}
Dabei entspricht $\symup{p_0}$ dem Normdruck, $\symup{T_0}$ der
Normtemperatur und $\symup{N}_\text{L}$ der Loschmidtschen Zahl.
Für den Brechungsindex $n$ gilt somit
\begin{equation}
      \symup{n} \left(p_0, T_0 \right) = 1 + \frac{f}{2} \: \symup{N}_\text{L}
      = 1 + \symup{Δ}n \left(p,p' \right) \frac{\symup{T}}{\symup{T_0}}
      \frac{p_0}{p-p'}\:.
      \label{eqn:brechindex}
\end{equation}
