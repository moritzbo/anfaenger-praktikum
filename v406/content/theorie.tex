\section{Theorie, nach \cite{Anleitung}}
\label{sec:theorie}
Für die Messung von Beugungs- und Interferenzeffekten werden neben köharenten
Wellen, Spalte mit Abmessungen kleiner der Wellenlänge der Quelle benötigt.

\subsection{Beugung}
Durch das Huygensche Prinzip, welches eine ebene Wellenfront aus Elementarwellen
zusammengesetzt darstellt und die Mittelung über viele Wellen kann das Wellenmodell
verwendet werden.
Es wird aufgrund der hohen Frequenz das Intensitätsbild $I\propto|B|^2$ gemessen.
\\
Aufgrund des Versuchsaufbaus wird die Fraunhoferbeugung zur Berechnung verwendet,
da die Abstände nicht zu Fresnels-Annahmen passen.
Die einfallenden Lichtstrahlen werden als ebene Wellen angenommen und der Abstand
zum Schirm als genügend groß.
Die Interferenzenerscheinungen entstehen durch Phasenunterschiede verschiedener
Einzelstrahlen. Die Phasendifferenz $δ$ zwischen diesen berechnet sich nach
\begin{equation}
      δ=\frac{2\symup{π}s}{λ}=\frac{2\symup{π}x\sin(φ)}{λ}
\end{equation}
Die Intensität an einem Ort ist das Quadrat der Summe aller einfallenden Amplituden,
dies wird zu einem Integral geführt. Die Breite des Spaltes ist $b$ und der\
Beobachtungswinkel ist $φ$.
Es ergibt sich
\begin{equation}
      I(φ)\propto B(φ)^2=A_0^2\,b^2 \left(\frac{λ}{\symup{π}b\sin(φ)}\right)^{\!2}
      \cdot\sin^2\!\left(\frac{\symup{π}b\sin(φ)}{λ}\right) \;.
      \label{eqn:einzelspalt}
\end{equation}
\subsection{Doppelspalt}
Für das Bild des Doppelspaltes werden zwei um $s$ verschobene Einzelspalte als
Grundlage verwendet. Die Einhüllende der Intensität ist deswegen ähnlich zu der
eines Einzelspaltes mit der entsprechenden Spaltbreite;
\begin{equation}
      I(φ)\propto B(φ)^2= 4\cos^2\!\left(\frac{\symup{π}s\sin(φ)}{λ}\right)
      \cdot\left(\frac{λ}{\symup{π}b\sin(φ)}\right)^{\!2}
      \cdot\sin^2\!\left(\frac{\symup{π}b\sin(φ)}{λ}\right) \;.
      \label{eqn:doppelspalt}
\end{equation}
\subsection{Fouriertransformation}
Wird eine Funktion die Geometrie der Öffnung beschreibt gebildet und ihre
Fouriertransformierte bestimmt, kann mit dieser auch die Verteilung der Minima
und Maxima, das Beugungsbild, beschrieben werden.
