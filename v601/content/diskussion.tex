\section{Diskussion}
\label{sec:Diskussion}
Beide Messergebnisse der Anregungsenergie $E_n$
\begin{align*}
    E_{n,1} &= \SI{5.4(3)}{\electronvolt} \\
    E_{n,2} &= \SI{5.22(7)}{\electronvolt}\:,
    \intertext{aus der Frank-Hertz-Kurve, liegen in der Nähe des Literaturwertes \cite{LitFH}}
    E_{n} &= \SI{4.9}{\electronvolt}\:.
\end{align*}
Die Abweichungen sind dadurch zu erklären, das
die Kalibrierung nicht ganz exakt ist, beziehungsweise die Maxima
nicht genau aus dem Diagramm entnommen werden konnten. Dies liegt zum einen an der
Minendicke des Stiftes zum anderen an der Einstellungsvariante der Spannung,
über Kippschalter und Zeigermessgerät. Außerdem war es ziemlich schwierig eine
konstante Temperatur einzustellen, was die Messergebnisse negativ beeinflusst.
\\
Für beide Methoden liegt die Abweichung zum Literaturwert unter
$\SI{10}{\percent}$. Aufgrund der oben aufgezählten Fehlerquellen ist die Abweichung für
eine Messung im akzeptablen Bereich.

Die elastischen Stöße müssen weniger aufgrund des auftretenden Energieverlustes
der $\symup{e}^-$, sondern wegen der Richtungsänderung beachtet werden.
\\
Die aufgenommene Kurve ähnelt sehr der Kurve aus der Anleitung. Somit konnte
die Ionisationsspannung bis auf weiteres berechnet werden. Die Unsicherheiten,
die die Messung beeinflussen sind zum einen die Temperaturschwankung, wie
bereits oben angemerkt, zum anderen die Skalierung der Graphen.
Außerdem hat die Bedienung des XY-Schreibers nicht dazu beigetragen
bessere Messungen durchzuführen.
