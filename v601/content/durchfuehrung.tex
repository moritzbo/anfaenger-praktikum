\section{Durchführung und Aufbau}
\label{sec:Aufbau}
Die Kurven werden mit einem XY-Schreiber aufgenommen,
Am X-Eingang liegen die Spannungen an, am Y-Eingang der Auffangstrom.
Die Heizung wird über ein sperates Gerät gesteuert.
Die Beschleunigungs- und Gegenspannung wird einzeln geregelt und über je ein
integriertes Zeigermessgerät angezeigt.
Der Auffangstrom wird mit einem Picoamperemeter gemessen und für
den X-Y-Schreiber in eine Spannung umgewandelt.
\\~\\
Zu Anfang jeder Messreihe wird die X-Achse in $\SI{2}{\volt}$-Schritten kalibriert.

In der ersten Messreihe wird zuerst der Auffangstrom in Abhängigkeit der
Gegenspannung aufgezeichnet, zunächst für Raumtemperatur,
dann mit einer Temperatur zwischen $\num{140} - \SI{160}{\celsius}$.
Daraus kann die integrale Energieverteilung bestimmt werden.

Für die zweite Messreihe werden bei verschiedenen konstanten Temperaturen Franck-Hertz-Kurven
genommen, im Temperaturbereich von $\num{160} - \SI{200}{\celsius}$.
$I_\text{A}$ wird in Abhängigkeit von $U_\text{B}$ gemessen.
Die Skalierung der Achsen wurde so angepasst, das möglichst viele Maxima
genommen werden können.
Die Temperatur wird in einem Intervall von $\SI{1}{\celsius}$ gehalten,
eine komplett konstante Temperatur ist nicht einstellbar.

In der dritten Messreihe, zur Bestimmung der Ionisierungsenergie wird eine
Gegenspannug von $U_\text{A} = \SI{-30}{\volt}$
angelegt. Die Temperatur liegt zwischen 100 und $\SI{110}{\celsius}$.
Es wird $I_\text{A}(U_\text{B})$ aufgetragen.
