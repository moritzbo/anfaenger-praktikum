\section{Diskussion}
\label{sec:Diskussion}
Die Bestimmung des Maximums bei feststehendem Kristall führt zu einer Abweichung
von $\incrementφ=\SI{0.4}{\degree}$.
Dieser Fehler wird bei der Beurteilung
der weiteren Ergebnisse nicht berücksichtigt werden, da er bei jedem Messwert auftritt.

Die charakteristischen Linien der Cu-Röhre weisen einen prozentualen Fehler
kleiner $\SI{1}{\percent}$ auf, dieser Fehler liegt vor allem an der 
Winkelauflösung des Geräts. Die Bestimmung der minimalen
Wellenlänge liefert einen Wert der um $\incrementλ_\%=\SI{0.8}{\percent}$ vom
Theoriewert abweicht.
Der Fehler der Energie liegt in der gleichen Größenordnung.
Die bestimmte minimale Energieauflösung ist vor allem von den gewählten
Winkelschritten abhängig, mit der verwendeten Schrittweite von
$\incrementθ=\SI{0.1}{\degree}$ wurde hier die kleinste verwendete Schrittweite
benutzt.

Die Abschirmkonstanten für Kupfer passen gut mit den Literaturwerten überein.
Die Abweichungen stammen aus der Energieauflösung und Winkelschrittweite.
\begin{align}
    σ_{1,m}&=3.398 &σ_{1,t}=\;3.411 \\
    σ_{2,m}&=13.968 &σ_{2,t}=13.121
\end{align}

Die Bestimmung der Rhydberg-Konstanten weist einen Fehler von
$\increment R_{\infty,\%} = \SI{11.47}{\percent}$ nach Formel \eqref{eqn:prozfehler} auf.
Das kann an den Fehlern der Abschirmkonstanten in Tabelle \ref{tab:absorptionmessung}
liegen. Außerdem geht der Fehler der Energieauflösung noch mit ein.
Der Fehler liegt aber für 4 Elemente mit je einer Messung im Rahmen
eines statistischen Fehlers.

Die Absorptionskanten des Quecksilbers sind nicht stark ausgebildet, deswegen
ist die Bestimmung der Absorptionswinkel schwierig.
Die Ablesefehler und 'grobe' Winkeleinstellungen sind hier dominierend
und führen zu der großen Abweichung von $\SI{32.4}{\percent}$.
\\
Verbesserungsmöglichkeiten um die Unsicherheiten zu minimieren wären zum einen
eine längere Integrationszeit und zum anderen feinere
Winkeleinstellungen zu wählen und mehrere Messungen durchzuführen.
