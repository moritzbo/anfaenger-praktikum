\section{Theorie}
\label{sec:theorie}
Die Theorie folgt der Versuchsanleitung, \cite{Anleitung}.
\subsection{Röntenstrahlung}
Bei Röntgenstrahlung handelt sich um hochfrequente elektromagnetische Wellen,
üblicherweise in einem Wellenlängenbereich von mehreren $\SI{10}{\pico\meter}$.
Elektronen werden durch den glühelektrischen Effekt
\footnote{siehe glühelektrischer Effekt, \cite{Anleitung504}} erzeugt,
und in Richtung der Anode beschleunigt. Sobald die Elektronen auf das
Coulomb-Potential des Kerns treffen, werden sie stark abgebremst.
Dieser Bremsvorgang führt dazu, dass Strahlung emittiert wird, welche sich als
Bremsstrahlung identifizieren lässt.
Es wird zwischen 2 Spektren unterschieden: Ein nahezu kontinuierliches Spektrum,
welches dadurch entsteht, dass die abgegebene Energie von der kompletten Abgabe
bis hin zu gequantelten Teilen reicht. Diese Art der Strahlung wird
Bremsstrahlung genannt. Die minimale auftretende Wellenlänge berechnet sich nach
\begin{equation}
  λ_\text{min} = \frac{\symup{h}\cdot\symup{c}}{e_0\:U}
  \label{eqn:lambdamin}
\end{equation}
und ergibt sich bei der vollständigen Abbremsung der Elektronen.
\\
Das diskrete Spektrum kommt zustande indem das Anodenmaterial
ionisiert wird. Die dadurch entstehende Lücke auf einer inneren Schale wird
durch ein weniger stark gebundenes Elektron wieder aufgefüllt. Die so entstehende
Energiedifferenz
\begin{equation}
  ΔΕ = ΔE_m\:-\:ΔE_n
\end{equation}
entspricht der Differenz der Schalenenergien.
Diese Energien sind diskret und äußern sich in Form von scharfen Peaks im Spektrum
und sind charakteristisch für jedes Element.
Die Bezeichnung der Linien folgt aus der Schale in welches das Elektron fällt
und aus der es kommt.
Das beobachtete Spektrum ist das sogenannte Emissionsspektrum.

\subsection{Abschirmung}
Da bei Atomen mit mehreren Elektronen die Valenzelektronen die einfallenden
Elektronen vom positiven Kern fernhalten, verringert sich die Bindungsenergie
des Elektrons der n-ten Schale auf
\begin{equation}
  E_n = -R_\infty\: z_\text{eff}^2 \cdot \frac{1}{n^2}\:.
  \label{eqn:bindE}
\end{equation}
Dabei ist $z_\text{eff} = z\:-\:σ$ die effektive Kernladungszahl und $σ$
die Abschirmkonstante der jeweiligen Schale. $R_\infty$ bezeichnet die Rydberg-Konstante mit
$R_\infty = \SI{13.6}{\electronvolt}$.
\\
Hiermit wird die Energie der $K_α$ Linie, nach Gleichung \eqref{eqn:bindE}, zu
\begin{equation}
  E_{\text{K}_α} = R_\infty \left( z\:-\:σ_1 \right)^2 \cdot \frac{1}{1^2}\:-\:
                 R_\infty \left( z\:-\:σ_2 \right)^2 \cdot \frac{1}{2^2}
  \label{eqn:kalpha}
\end{equation}
berechnet.
\\
\subsection{Absorptionsspektrum}
Besitzt die Röntgenstrahlung eine Energie kleiner als
$\SI{1}{\mega\electronvolt}$, sind bei der Absorption der Compton-Effekt und der
photoelektrische Effekt dominant.
Steigt die Photonenenergie über die Bindungsenergie der nächst inneren Schale,
steigt der Absorptionskoeffizient schlagartig an. Dieses rapide Ansteigen wird
als Absorptionskante bezeichnet und ihre Lage berechnet sich nach
\begin{equation}
  \symup{h}ν_\text{abs} = E_n\:-\:E_\infty\:.
  \label{eqn:enu}
\end{equation}
Aufgrund der Feinstruktur der Absorptionskanten muss die Bindungsenergie
$E_\text{n,j}$ mit der Sommerfeld'schen Feinstrukturformel
aus \cite{Anleitung}
berechnet werden.
\\
Die Abschirmkonstante $σ_L$ beschreibt, wie
stark das Coulombpotential des Kerns durch die gebundenen Elektronen
abgeschwächt wird.
Sie berechnet sich nach
\begin{equation}
  σ_L = Z\:-\:\left(
                \frac{4}{α} \sqrt{ \frac{\increment E_\text{L}}{R_\infty}}
                -\frac{5\increment E_\text{L}}{R_\infty}\right)^{\frac{1}{2}}
        \left(1 + \frac{19}{32}α^2\frac{\increment E_L}{R_\infty}\right)^{\frac{1}{2}}\:,
        \label{eqn:sigmaL}
\end{equation}
mit der Sommerfeld'schen Feinstrukturkonstanten $α$ und der Energiedifferenz der
Schalen $\increment E_L$.
Die Energie der Röntgenbremsstrahlung wird experimentell mittels der
Braggreflektion bestimmt. Dabei wird das Röntgenlicht an einem Gitter gebeugt
und interferiert. Bei einem bestimmten Winkel, dem sogenannten Glanzwinkel,
tritt konstruktive Interferenz auf, welche nach
\begin{align}
  E &= \frac{\symup{h}\symup{c}}{λ}
  \shortintertext{und}
   nλ &= 2d\sin{θ}
  \shortintertext{die Energie E gemäß}
  E &= \frac{\symup{h}\,\symup{c}}{2\,d\,\symup{e}_0\,\sin(θ)}
\label{eqn:BraggE}
\end{align}
besitzt. Die Abbildung \ref{fig:braggy} stellt dies dar.
Für die K-Linien können die Abschirmkonstanten $σ_1$ und $σ_2$ aus der
Energiedifferenz $\increment E_\text{K} = E_{K_α}\:-\:E_{K_β}$ der beiden Linien
errechnet werden.

\newpage

\begin{figure}
    \caption{Braggreflektion an einem Kristall.\protect\footnotemark}
    \label{fig:braggy}
    \begin{center}
      \begin{circuitikz}
        \draw (-0.3,0) -- (5.3,0);
        \draw (-0.3,1) -- (5.3,1);
        \draw (-0.3,2) -- (5.3,2);
        \draw[<->, very thick] (5,1) -- (5,2);
        \node at (5.3,1.5) {$\text{d}$};
        \coordinate (a) at (0,0);
        \coordinate (b) at (1,0);
        \coordinate (c) at (2,0);
        \coordinate (d) at (3,0);
        \coordinate (e) at (4,0);
        \coordinate (f) at (5,0);
        \coordinate (g) at (0,1);
        \coordinate (h) at (1,1);
        \coordinate (i) at (2,1);
        \coordinate (j) at (3,1);
        \coordinate (k) at (4,1);
        \coordinate (l) at (5,1);
        \coordinate (m) at (0,2);
        \coordinate (n) at (1,2);
        \coordinate (o) at (2,2);
        \coordinate (p) at (3,2);
        \coordinate (q) at (4,2);
        \coordinate (r) at (5,2);

        \foreach \point in {a,b,c,d,e,f,g,h,i,j,k,l,m,n,o,p,q,r} %schwarze punkte
        \fill [black,opacity=1] (\point) circle (2pt);
        \draw[->,very thick] (0,4) -- (2.5,2); %pfeile
        \draw[->,very thick] (2.5,2) -- (5,4);
        \node at (3.2,2.3) {$θ$};
        \node at (1.8,2.3) {$θ$};
        \draw (3.5,2) arc (0:38.65980825:1);
        \draw (1.5,2) arc (180:180-38.65980825:1);
      \end{circuitikz}
    \end{center}
\end{figure}
\footnotetext{Die Skizze wurde mit Tikz erstellt.}
