\section{Theorie}
\label{sec:theorie}
\subsection{Allgemeines}
Brückenschaltungen werden im allgemeinen dafür verwendet,
dass sie die Auflösung von Messungen erhöhen.
Das wird mittels der Brückenspannung $\symup{U}_\text{Br}$
verrichtet, welche im Folgenden Verwendung findet.
Mit Hilfe von Brückenschaltungen können Größen, welche diskret als
elektrische Widerstände fungieren, vermessen werden.
\\
\\
Um nun zum Beispiel einen Widerstand zu vermessen, ist es das Ziel die
Brückenspannung zu minimieren.
Dies lässt sich mittels der \emph{Kirchhoff'schen Regeln} erreichen.
\subsection{Kirchhoff'sche Regeln}
Das erste Gesetz ist die \emph{Knotenregel},
\begin{equation}
  \sum_m \symbf{I}_\text{m} \: = \: 0
\end{equation}
welche besagt, dass die Summe der Ströme, die in einen Knoten hineinfließen
gleich der Summe der hinausfließenden Ströme ist.
Die zweite Regel, die \emph{Maschenregel}, besagt,
\begin{equation}
  \sum_n \symup{E}_\text{n} \: = \: \sum_n \symbf{I}_\text{n} \symup{R}_\text{n}
\end{equation}
dass die Summe der EMK\footnote{elektromotorischen Kräfte} innerhalb
eines geschlossenen Kreises, einer Masche, gleich der Summe der Produkte aus
$\symbf{I}$ und $\symup{R}$ ist.
Wird nun die Maschenregeln für eine Brückenschaltung ausgeführt,
folgt ein Ausdruck für die Spannung $\symup{U}$
\begin{equation}
      \symup{U} = \frac{\symup{R}_2 \symup{R}_3 - \symup{R}_1 \symup{R}_4}
      {(\symup{R}_3 + \symup{R}_4)
      (\symup{R}_1 + \symup{R}_2)} \symup{U}_\text{s} \,.
      \label{eqn:quotient}
\end{equation}
Die Schaltskizzen der verschiedenen Brückenschaltungen sind im Anhang zu finden.
Daraus kann durch geschicktes umformen ein Ausdruck für die
Brückenspannung gefunden werden, welcher nur abhängig von den Bausteinen
der Schaltung ist.
\begin{equation}
  \symup{R}_1 \symup{R}_4 = \symup{R}_2 \symup{R}_3
\end{equation}
wird auch Abgleichbedingung genannt, woraus nun der
zu vermessende Widerstand bestimmt werden kann.
Analog funktioniert die Abgleichbedingung auch für komplexe Widerstände,
hier werden die zusammengesetzten Impedanzen Z verwendet.
\begin{equation}
  \symup{Z}_1 \symup{Z}_4 = \symup{Z}_2 \symup{Z}_3
\end{equation}
Dies liefert, mit Koeffizientenvergleich, zwei neue Gleichungen
\begin{align}
  \symup{X}_1\symup{X}_4 - \symup{Y}_1\symup{Y}_4 &=
  \symup{X}_2\symup{X}_3 - \symup{Y}_2\symup{Y}_3
  \label{eqn:abgleich1} \\
  \symup{X}_1\symup{Y}_4 - \symup{X}_4\symup{Y}_1 &=
  \symup{X}_2\symup{Y}_3 - \symup{X}_3\symup{Y}_2.
  \label{eqn:abgleich2}
\end{align}

\subsection{Bestimmung der Unbekannten Größen}

Für die Bestimmung des unbekannten Widerstandes der Wheatstonebrücke,
wird die Abgleichbedingung nach $\symup{R}_\text{x} / \symup{R}_\text{1}$ umgestellt.
Es folgt
\begin{equation}
    \symup{R}_\text{x} = \symup{R}_2\frac{\symup{R}_3}{\symup{R}_4}\,,
    \label{eqn:wheat}
\end{equation}
Für die Genauigkeit ist vor allem der Quotient aus $\symup{R}_3/\symup{R}_4$ von
Bedeutung, dass heißt wie genau das Potentiometer eingestellt ist.
\\
Bei der Kapazitätsmessbrücke besteht der reale Kondensator aus einem
Kondensator und einem Widerstand welche in Reihe geschaltet sind.
\begin{figure}
      \begin{center}
            \begin{circuitikz}
                  \draw[thick] (0,0) to[resistor] node[midway, below] {$\symup{R}$} (2,0)
                  to[capacitor] node[midway, below] {$\symup{C}$} (4,0);
            \end{circuitikz}
      \end{center}
\end{figure}
\\
Die Summe der Impedanzen Z liefert
\begin{equation}
  \symup{Z}_\text{real} = \symup{R} - \frac{\symup{i}}{\omega \symup{C}}\,.
\end{equation}
Die Abgleichbedingung liefert dann:
\begin{align}
  \symup{R}_\text{x} &= \symup{R}_2 \frac{\symup{R}_3}{\symup{R}_4}
  \label{eqn:kapar} \\
  \symup{C}_\text{x} &= \symup{C}_2 \frac{\symup{R}_4}{\symup{R}_3} .
  \label{eqn:kapac}
\end{align}
\\
Analog gilt auch für die Induktivitätsmessbrücke,
dass den verlustbehafteten Induktivitäten der Gestalt
\begin{figure}
      \begin{center}
            \begin{circuitikz}
                  \draw[thick] (0,0) to[resistor] node[midway, below] {$\symup{R}$} (2,0)
                        to[cute inductor] node[midway, below] {$\symup{L}$} (4,0);
                  \end{circuitikz}
            \end{center}
\end{figure}
\\
eine Impedanz
\begin{equation}
  \symup{Z}_\text{real} = \symup{R} + \symup{i} \omega \symup{L}
\end{equation}
zugeodnet wird.
Damit berechnen sich die Induktivitäten und zugehörigen Widerstände zu
\begin{align}
  \symup{R}_\text{x} &= \symup{R}_2 \frac{\symup{R}_3}{\symup{R}_4}
  \label{eqn:indur} \\
  \symup{L}_\text{x} &= \symup{L}_2 \frac{\symup{R}_3}{\symup{R}_4} .
  \label{eqn:indul}
\end{align}
Da solche Schaltungen im niedrigfrequenten Bereich schwierig realisierbar sind,
werden häufig sogenannte Maxwell-Brücken verwendet.
\\
Bei der Maxwell-Brücke werden Potentiometer als regelbare Widerstände benutzt,
$\symup{R}_2$ und $\symup{C}_4$ sind bekannt und konstant. Sie können, durch wechseln, variiert werden.
Die zusammengesetzte Impedanz hat die Form
\begin{align}
  \symup{Z}_1 &= \symup{R}_\text{x} + \symup{i}\omega\symup{L}_\text{x} \\
\shortintertext{und}
  \frac{1}{\symup{Z}_4} &= \frac{1}{\symup{R}_4} + \symup{i}\omega\symup{C}_4 .
\end{align}
Aus den Abgleichbedingungen \eqref{eqn:abgleich1} und \eqref{eqn:abgleich2}
resultiert für $\symup{R}_\text{x}$ und $\symup{L}_\text{x}$
\begin{align}
  \symup{R}_\text{x} &= \frac{\symup{R}_2 \symup{R}_3}{\symup{R}_4},
  \label{eqn:maxr} \\
\shortintertext{sowie}
  \symup{L}_\text{x} &= \symup{R}_2 \symup{R}_3 \symup{C}_4.
  \label{eqn:maxl}
\end{align}
\\
All diese Brücken sind frequenzunabhängig.
Für die frequenzabhängige Brückenspannung, der Wien-Robinson-Brücke,
$\symup{U}_\text{Br}$ ergeben sich vier Widerstandoperatoren $\symup{Z}$
\begin{align}
  \symup{Z}_1 &= 2\symup{R'} & \symup{Z}_2 &= \symup{R'}\\
  \symup{Z}_3 &= \symup{R} + \frac{1}{\symup{i}\omega\symup{C}}
  & \symup{Z}_4 &= \frac{\symup{R}}{1 + \symup{i} \omega \symup{RC}} .
\end{align}
Darauf wird Gleichung \eqref{eqn:quotient} angewandt, wobei alle Widerstände
$\symup{R}_\text{i}$ gegen Impedanzen $\symup{Z}_\text{i}$ ausgetauscht werden.
Daraus erhält man einen Ausdruck für
${\symup{U}_\text{Br}}/{\symup{U}_\text{s}}$. Das Betragsquadrat ergibt dann
\begin{equation}
  \left| \frac{\symup{U}_\text{Br}}{\symup{U}_\text{s}} \right|^2 =
  \frac{\left(\omega^2 \symup{R}^2 \symup{C}^2 -1 \right)^2}
      {9\left\{ \left(1 - {\omega}^2 \symup{R}^2 \symup{C}^2 \right)^2 +
      9 {\omega}^2 \symup{R}^2 \symup{C}^2 \right\}} .
      \label{eqn:wrfrequenz}
\end{equation}
Dies liefert den Zusammenhang für die Eigenkreisfrequenz,
bei welcher die Brückenspannung verschwindet,
\begin{equation}
  ω_0 = 1/\symup{RC} .
\end{equation}
Durch eine Substitution mit $\Omega = ω\symup{RC}$ ergibt sich
\begin{equation}
\label{eqn:UbrUs}
  \left| \frac{\symup{U}_\text{Br}}{\symup{U}_\text{s}} \right|^2 =
  \frac{1}{9}
  \frac{\left(\Omega^2 - 1 \right)^2}{\left( 1 - \Omega^2 \right)^2
  +9\Omega^2} .
\end{equation}
\\
Die Wien-Robinson-Brücken wird also als Filter benutzt, welcher Frequenzen um
$\omega_0$ unterdrückt, also ein Sperrfilter ist.
\\
Der Klirrfaktor $\symup{k}$ wird auch Verzerrungsgehalt genannt und ist ein Maß dafür,
wie stark die Oberwellen des anliegenden Sinussignals sind.
Er berechnet sich nach
\begin{equation}
  \symup{k} = \frac{\sqrt{\sum_{i=2}^\text{n}
  \symup{U}_\text{i}^2}}{\symup{U}_1} .
  \label{eqn:klirr}
\end{equation}
Wobei $\symup{U}_1$ die Amplitude der Grundwelle ist und die
$\symup{U}_\text{i}$ die Amplituden der i-ten Oberwelle bei $i\cdotν$.
