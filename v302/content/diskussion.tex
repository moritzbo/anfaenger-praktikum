\section{Diskussion}
\label{sec:Diskussion}
Bei allen Messungen ist es nicht möglich die Brückenspannung auf $\SI{0}{\volt}$
zu setzen. Das liegt vor allem an der Skalierung des Potentiometers.
Dies wird mit der mehrfachen Messung unter Variation verschiedener Bauteile versucht
zu kompensieren.
\\
Da die Gauß-Fehler bei der Wheatstone-Brücke dominieren ist ein statistischer Fehler nicht anzunehmen.
Ein systematischer Fehler ist dennoch möglich.
\\
Für die Messung der idealen Kapazität liegt der Fehler bei diesen Messungen ebenfalls in der Größenordnungs wie die Fehler
der Bauteile. Die Annahme, dass es sich um hochwertige Kondensatoren handelt ist folglich korrekt.
Die Messung der realen Kapazität weißt einen Fehler auf, der im Rahmen einer akzeptablen Messungenauigkeit liegt,
vor allem bei nur drei genommen und zwei gemittelten Werten.
Die gemessene Amplitude der Brückenspannung betrug hier noch einige $\SI{10}{\milli\volt}$,
konnte aber durch Verstellen der Widerstände nicht verkleinert werden.
\\
Die Fehler der Induktionsbrücke sind größer als bei den vorangegegangen Messreihen,
da der verstellbare Widerstand $R_2$ einen größeren Fehler
als die vorher verwendeten festen Widerstände hat.
Der Widerstand $R_2$ musste in der zweiten Messung so groß gewählt werden,
damit die Brückenspannung nicht mehr einige $\si{\milli\volt}$ beträgt.
Insgesamt sind die Fehler aber in einer akzeptablen Größenordnung.
\\
Die Ergebnisse der Induktivitätsbrücke unterscheiden sich vor allem bei der Spule
stark von denen der Maxwellbrücke. Da bei der Maxwellbrücke $\sfrac{2}{3}$ der Werte
aufgrund ihrer Bestimmungsmöglichkeit wegfallen, ist davon auszugehen, dass auch
der verbleibende Wert in der Bestimmung nicht gut gelungen ist, was bei der
Induktivitätsbrücke jedoch nicht so ist.
\\\\
Die Ergebnisse der Messung der Amplitude der Maxwellbrücke liegen in guter Näherung
zur Theoriekurve. Lediglich die Abweichung bei $ω_0$ ist zu nennen, da hier der
Klirr-Faktor zu beachten ist.
\\\\
Wie bei der Frequenzabhängigkeit der Maxwellbrücke zu sehen ist, gehen die Messwerte nahe an 0.
Der Klirrfaktor ist demnach mit $\SI{5.06}{\percent}$ gut bestimmt, da dieser Wert für die Anteile der Oberwellen am gesamten Signal steht.
Eben diese sind aber nicht vom Sperrfilter betroffen und führen zur überbleibenden Brückenspannung. 
