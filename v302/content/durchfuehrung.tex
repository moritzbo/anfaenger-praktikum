\section{Durchführung und Aufbau}
\label{sec:Durchführung}
Die Schaltpläne sind im Anhang zu finden.
\\
Zunächst wird die Wheatstoneschaltung gemäß Abbildung \ref{fig:wheatbruecke}
verwendet, um den Widerstand (hier: $\symup{R}_\text{x}$ an der Position $R_1$) zu bestimmen. Als
$\symup{R}_\text{x}$ werden die Widerstände \enquote{Wert 12} und \enquote{Wert 13} verwendet. Der Widerstandsquotient aus
$\symup{R}_\text{3}\:/\:\symup{R}_\text{4}$ entspricht einem Potentiometer
mit einer Größe von$\SI{1}{\kilo\ohm}$. Die Skala ist in 1000 Schritte eingeteilt.
Zur Bestimmung von $\symup{R}_\text{x}$ wird die Messung mit drei
unterschiedlichen Widerständen $\symup{R}_2$ durchgeführt.
Das Potentiometer wird so eingestellt, dass die auf dem Oszilloskop
dargestellte Brückenspannung $U_\text{Br}$, minimal wird.
\\
Die Kapazitätsmessbrücke wird wie in Abbildung \ref{fig:kapabruecke}
aufgebaut. Es werden zum einen ein realer Kondensator \enquote{Wert 8} und zum
anderen zwei ideale Kondensatoren, \enquote{Wert 1} und \enquote{Wert 3}, an der Position $\symup{C}_\text{x}$ vermessen.
Der Widerstand $\symup{R}_\text{2}$ wird nur beim realen Kondensator mit eingebaut, da sonst
hochwertige Kondesatoren verwendet werden.
Es wird wie bei der Wheatstoneschaltung verfahren und der Wert des
Potentiometers bei minimaler Brückenspannung genommen.
Für den realen Kondensator wird zudem $R_2$ wechselweise mit dem Potentiometer verändert.
\\
Mit der Induktivitätsmessbrücke wird eine unbekannte Spule vermessen.
Als unbekannte Spule wird die mit der Aufschrift \enquote{Wert 17} verwendet.
Dabei werden jeweils drei unterschiedliche Induktivitäten für $\symup{L}_2$
verwendet. Auch hier wird $R_2$ im Wechsel mit dem Potentiometer variiert.
\\
Danach wird die Spule \enquote{Wert 17} mit der Maxwell-Brücke
vermessen. Es wird dafür jeweils die Induktivität $\symup{L}_4$ variiert und
an den Potentiometern drei Widerstandswerte entnommen.
\\
Die Wien-Robinson-Brücke wird wie in Abbildung
\ref{fig:wrbruecke} aufgebaut um unter Variation der Frequenz $ν$ das
Verhalten der Brückenspannung zu beobachten.
\\
\\
Die verwendeten Widerstände, Induktivitäten und Kondensatoren werden in der
Auswertung in den jeweiligen Tabellen aufgeführt.
