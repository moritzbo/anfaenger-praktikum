\section{Diskussion}
\label{sec:Diskussion}
Die erste Messung der Schallgeschwindigkeit mit dem Puls-Echo-Verfahren an nur
einem Zylinder liefert ein Ergebnis mit nur $\SI{0.05}{\percent}$ Abweichung.
Die Koppelschicht mit Wasser hat bei dieser Messung also keinen großen Einfluss.

Die Dämpfungskonstante $α$ ist bei uns negativ, was im Exponenten zu einem Abfall
führt. Das ist sinnvoll.
Der Vergleich mit der Dämpfungskonstanten, einer anderen Messung am gleichen Objekt,
von $α = 0,017$ \cite{daempf} zeigt, dass unser Wert von $α = \num{-0.023(4)}$ eine Abweichung von
2 $σ$-Umgebungen hat. Da der erste Wert aus einer Messung stammt, ist dieser auch mit
einem unbekannten Fehler behaftet.
Die Messungen mit den gestapelten Zylindern mussten
ausgelassen werden, da die Reflexionen nicht eindeutig zugeordnet werden konnten.

Die Schallgeschwindigkeitsbestimmung mit diesen Werten ergibt einen Wert von\\
$c = \SI{2732(4)}{\meter\per\second}$, der Literaturwert liegt somit in der
1-$σ$-Umgebung unseres Wertes.

Auch das Durchschallungsverfahren liefert eine Schallgeschwindigkeit nahe des
Literaturwertes, hier war das Auftragen des Wassers als Koppelschicht
aufgrund der Schwerkraft schwieriger, was eine mögliche systematische Fehlerquelle ist.

Die Plattendicken, sowie die Zylinderhöhe stimmen gut mit den gemessenen Werten
überein. Hier ist das Wasser als Koppelmittel wieder der größte Fehler.

Die Messwerte der Messung mit dem Auge sehen zunächst zu groß aus, da das Augenmodell
jedoch im Maßstab 1:3 war, passen diese doch.
