\section{Diskussion}
\label{sec:Diskussion}
Bezüglich der Wärmekapazitäten lässt sich sagen, dass die Messungen, bis auf
die von Blei, ziemlich fehlerbehaftet sind, wie es auch aus den
Messergebnissen ausgelesen werden kann. Die Bestimmung der Kalorimeter weißt
zudem diverse Fehlerquellen auf. Zum einen kann die Temperatur nur mit einem
Thermometer endlich genau bestimmt werden und zum anderen sind alle benutzten
Gefäße nicht vollständig thermisch abgedichtet, sodass es zu dissipativen
Verlusten kommt. Dies wird nochmals durch die Abweichungen der beiden
Kalorimeterwerte zueinander unterstrichen.
Einen analogen Argumentationsgang kann für die Wärmekapazitäten der
verschiedenen Stoffe verwendet werden. Zusätzlich ist die Messung der Innentemperatur
der Stoffe ein Problem, da nur die Temperatur des den Stoff umgebenden Wassers
gemessen werden kann. Da die Materialien alle unterschiedlich gut Wärme
speichern können, geben manche Stoffe die Wärme besser ab und manche schlechter.
\begin{table}
  \centering
  \caption{Wärmeleitfähigkeit $\lambda$ der Messproben\cite{wärmeleitung}.}
  \label{tab:leitfähigkeit}
  \begin{tabular}{c S[table-format=3.0]}
    \toprule
    {Material} & {$λ \:/\: [\si{\watt\per\meter\per\kelvin}]$} \\
    \midrule
    Blei & 35 \\
    Graphit & 140 \\
    Kupfer & 380 \\
    \bottomrule
  \end{tabular}
\end{table}
\\
Umso größer $\lambda$ desto größer ist der \enquote{Wärmeübertrag} pro
Zeitintervall. Das heißt, umso besser sich die Atome in einem Material bewegen
können, desto größer ist die Wärmeabgabe. Kupfer überträgt Wärme sehr gut und
Blei eher schlechter. Deswegen kann die gemessene Wassertemperatur nicht
sinnvoll benutzt werden, um die Innentemperatur der Materialien zu bestimmen
\cite{wärmeleitung}.
