\section{Theorie}
\label{sec:theorie}

\subsection{Wärmekapazität}
Verändert sich die Temperatur eines Körpers um $\symup{Δ} T$, ohne geleistete Arbeit,
nimmt dieser die Wärmemenge
\begin{equation}
      \symup{Δ} Q = m \symup{c} \symup{Δ} T
      \label{eqn:formel1}
\end{equation}
auf. $m$ ist hierbei die Masse des Körpers und $c$ die Wärmekapazität.
\\
Die Wärmekapazität kann mit der Masse normiert werden.
Es folgt die Atom- oder Molwärme $C$, die angibt, welche Wärmemenge $\symup{d}Q$
ein Grammatom des jeweiligen Stoffes um die Temperatur $\symup{d}T$ verändert.
Wichtig ist hierbei welche Größe konstant bleibt.
\\
Wird das Volumen des Körpers konstant gehalten folgt mit Gleichung \ref{eqn:formel1}
\begin{align}
      \symup{C}_\text{V} &= \left(\frac{\symup{d}Q}{\symup{d}T}\right)_\text{V} \:.
      \label{eqn:formel2}
      \intertext{Mit der getroffenen Annahme des konstanten Volumens kann der erste Hauptsatz der Thermodynamik \cite{Anleitung}}
      \symup{d}U = \symup{d}Q + \symup{d}A &= \symup{d}Q - p\symup{d}V = \symup{d}Q - 0
      \label{eqn:formel3}
      \shortintertext{eingesetzt werden;}
      \symup{C}_\text{V} &= \left(\frac{\symup{d}U}{\symup{d}T}\right)_\text{V} \:.
      \label{eqn:formel4}
\end{align}

\subsection{Gesetz von Dulong-Petit}
Die grundlegende Annahme des Dulong-Petitschen Gesetzes ist die Unabhängigkeit
der Atomwärme von den chemischen Eigenschaften des Stoffes. Jedoch nur wenn im
festen Aggregatzustand das Volumen konstant ist, gilt
\begin{align}
      |\symup{C}_\text{V}| &= 3 \symup{R}
      \label{eqn:dulongpetit1}
      \shortintertext{mit der allgemeinen Gaskonstanten \cite{Anleitung}}
      \symup{R} &= \SI{8.314}{\joule\per\mol\per\kelvin} \:.
\end{align}
Die Herleitung erfolgt über die Mittelung der kinetischen und potentiellen Energie der Atome,
unter besonderer Betrachtung der Freiheitsgrade.
Mit Hilfe des eindimensionalen harmonischen Oszillators folgt
die gemittelte innere Energie als
\begin{equation}
      \left<u\right> = 2 \left<E_\text{kin}\right> \:.
\end{equation}
Mit dem Äquiparitionstheorem
\begin{equation}
      \left<E_\text{kin}\right> = \frac{1}{2} \symup{k}T \:,
\end{equation}
nachdem ein Atom pro Freiheitsgrad die oben gegebene gemittelte Energie besitzt,
wenn es im thermischen Gleihgewicht mit der absoluten Temperatur $T$ und der Umgebung ist.
Wird dies auf ein Mol und 3 Bewegungsfreiheitsgrade erweitert folgt Gleichung
\eqref{eqn:dulongpetit1}.

\subsection{Quantenmechanische Betrachtung}
Für hohe Temperaturen stimmt das oben gezeigte Dulong-Petit Gesetz mit den
experimentell festgestellten Werten überein.
Für tiefe Temperaturen stimmen Theorie und Experiment nicht mehr überein.
Dies liegt an der Quantelung der Energie, sodass die Energien der schwingenden Atome
nicht kontinuierlich verändert werden können. Die diskrete Teilung führt zum Abstand
\begin{equation}
      \symup{Δ}u = n \symup{\hbar} ω
\end{equation}
der Energieniveaus. $\symup{\hbar}$ ist das Planksche Wirkungsquantum und $n \in \symbb{N}$.
Die Wahrscheinlichkeiten der einzelnen Niveus sind durch
die Boltzmann-Verteilung gegeben. Die Herleitung nach \cite{Anleitung} führt zu
\begin{equation}
      \left<U_\text{qu}\right> = 3 N_\text{L} \left<u_\text{qu}\right> = 3 N_\text{L} \frac{\symup{\hbar} ω}{\text{exp}\left(\frac{\symup{\hbar}ω}{\symup{k}T}\right) - 1} \:.
      \label{eqn:uqu1}
\end{equation}
$N_\text{L} = \SI{6.02e23}{\per\mol}$ ist die Loschmidtsche Zahl, \cite{Anleitung}.
Für hohe Temperaturen nähert sich dieser Ausdruck an
\begin{equation}
      \left<U_\text{kl}\right> = 3 \symup{R} T
\end{equation}
an, somit auch an Gleichung \eqref{eqn:dulongpetit1}.

\subsection{Bestimmung der Wärmekapazität}
Im Experiment ist es einfacher den Druck statt des Volumens konstant zu halten.
Wichtig ist jetzt der Zusammenhang zwischen den Wärmekapazitäten bei konstantem Druck oder Volumen
\begin{equation}
      \symup{C}_\text{p} - \symup{C}_\text{V} = 9 \symup{α}^2 \symup{κ} \symup{V}_0 T \:.
      \label{eqn:vergleich}
\end{equation}
Dabei ist $\symup{C}_\text{p}$ der experimentell bestimmte Wert und $\symup{C}_\text{V}$ der Theoriewert,
entweder nach Dulong-Petit oder der Quantenmechanik.
Mit dieser Voraussetzung kann die spezifische Wärmekapazität mit einem
Mischungskalorimeter bestimmt werden.
Betrachtet werden jetzt nur noch Wärmemengen $Q_{\!i}$.
Die Bedeutungen der Indizes und Variablen sind am Ende des Kapitels aufgeführt.
\begin{equation}
      Q_1 = \symup{c}_\text{k} m_\text{k} (T_\text{k} - T_\text{m})
\end{equation}
ist die Wärmemenge die ein Körper $k$ abgibt, wenn er von der Temperatur $T_\text{k}$
auf $T_\text{m}$ abkühlt.
Geschieht dies in einem wassergefüllten Kalorimeter, wird die Wärmemenge
\begin{equation}
      Q_2 = (\symup{c}_\text{w} m_\text{w} + \symup{c}_\text{g} m_\text{g})(T_\text{m} - T_\text{w})
\end{equation}
aufgenommen. Mit der Annahme eines abgeschlossenen Systems kann $Q_1 = Q_2$ gesetzt werden.
Es folgt
\begin{equation}
      \symup{c}_\text{k} = \frac{(\symup{c}_\text{w} m_\text{w} + \symup{c}_\text{g} m_\text{g})(T_\text{m} - T_\text{w})}{m_\text{k}(T_\text{k} - T_\text{m})} \:.
      \label{eqn:wärmekapazität}
\end{equation}
Zur Bestimmung von $\symup{c}_\text{g} m_\text{g}$ wird als Körper Wasser genommen.
Analog zur Rechnung oben folgt
\begin{equation}
      \symup{c}_\text{g} m_\text{g} = \frac{\symup{c}_\text{w} m_\text{h}(T_\text{h} - T_\text{m}) - \symup{c}_\text{w} m_\text{c}(T_\text{m} - T_\text{c})}{(T_\text{m} - T_\text{c})} \:.
      \label{eqn:kaloriemeter}
\end{equation}
\begin{align*}
      &\text{Wert:} &Q &= \text{Wärmemenge} \\
      &             &\symup{c} &= \text{Wärmekapazität} \\
      &             &m &= \text{Masse} \\
      &             &T &= \text{Temperatur} \\
      &\text{Index:} &\symup{k} &= \text{Körper} \\
      &              &\symup{w} &= \text{Wasser im Kalorimeter} \\
      &              &\symup{g} &= \text{Kalorimeter} \\
      &              &\symup{m} &= \text{Mischtemperatur im Kalorimeter} \\
      &              &\symup{h} &= \text{heiße Wassermenge} \\
      &              &\symup{c} &= \text{kalte Wassermenge}
\end{align*}
