\section{Theorie}
\label{sec:theorie}
\subsection{Relaxation}
Als Relaxation bezeichnet man die Erscheinungen die auftreten, wenn
ein System aus seinem Ausgangszustand entfernt wird, es dann aber
nicht-oszillatorisch wieder dahin zurückkehrt. Dieser Vorgang tritt
beispielsweise bei der Auf- und Entladung eines Kondensators auf.
\cite{Anleitung}


\subsection{Entladung}

Die Ladung Q auf einem Kondensator, kann mit
\begin{align}
  Q &= U_\text{C} \symup{C}
  \label{eqn:kondensator1}
  \intertext{bestimmt werden. Beim Aufladevorgang sind die Randbedingungen}
  Q(0) &= \SI{0}{\ampere} \\
  Q(\infty) &= \symup{C} U_\text{C}
  \intertext{gegeben. Mit}
  \frac{\symup{d}Q}{\symup{d}t} &= -I
  \label{eqn:kondensator2}
  \shortintertext{und}
  \frac{U_\text{C}}{\symup{R}} &= I,
  \label{eqn:kondensator3}
  \intertext{sowie \eqref{eqn:kondensator1}, folgt}
  \symup{d}Q &= \int_0^\infty -\frac{Q}{\symup{RC}} \, \symup{d}t.
  \intertext{Das Ausführen der Integration führt zu:}
  Q(t) &= Q(0) \: \text{exp}\left(\frac{-t}{\symup{RC}}\right).
  \intertext{Für die Auswertung \ref{sec:auswertung1} wird nun nach $t$ umgestellt:}
  -\frac{1}{\symup{RC}} \, t &= \log{\left(\frac{U_\text{C}(t)}{\symup{U}_0}\right)}.
  \label{eqn:aformel}
\end{align}


\subsection{Relaxation bei periodischer Anregung}

Die angelegte Anregung wird als
\begin{equation}
  U(t) = \symup{U}_0 \cos{\symup{ω}t}
\end{equation}
angenommen. Wenn $ω < \! \!<  \sfrac{1}{\symup{RC}}$ ist, kann der Kondensator sich
schnell genung auf- und entladen, sodass $U_\text{C}(t) \approx U(t)$ gilt.
Wird die Frequenz größer hängt die Kondensatorspannung der Erregerspannung hinterher.

Gesucht sind jetzt die Phase und Amplitude von $U_\text{C}$.
Nach dem 2. Kirchhoff'schen Gesetz ist
\begin{equation}
  U(t) = U_\text{R}(t) + U_\text{C}(t).
  \label{eqn:kirchhoff}
\end{equation}
Als Ansatzfunktion für $U_\text{C}$ wählt man
\begin{equation}
  U_\text{C}(t) = A(ω) \cos(ωt + φ).
\end{equation}
Die mit den Formeln \eqref{eqn:kondensator1}, \eqref{eqn:kondensator2} und \eqref{eqn:kondensator3}
vereinfachte Formel,
\begin{equation}
  U_0 \cos(ωt) = -Aω\symup{RC} \sin(ωt+φ) + A(ω) \cos(ωt+φ),
  \label{eqn:relax1}
\end{equation}
wird nach $φ$ umgestellt. Dafür werden, geschickt gewählte, explizite Werte,
zB für $ωt$, eingesetzt.
\begin{equation}
  φ(ω) = \arctan(-ω\symup{RC})
  \label{eqn:cformel}
\end{equation}
Für die Bestimmmung der Amplitude A wird \eqref{eqn:relax1} ebenfalls vereinfacht und es folgt:
\begin{equation}
  A(ω) = \frac{U_0}{\sqrt{1+ω^2\symup{R}^2\symup{C}^2}}.
  \label{eqn:bformel}
\end{equation}


\subsection{Integration mittels RC-Kreis}

Der RC-Kreis kann als Integrator für eine Spannung dienen, sprich
\begin{equation}
  U_\text{C} \propto \int U(t) \, \symup{d}t,
\end{equation}
wenn $ω > \! \! > \sfrac{1}{\symup{RC}}$ gilt.
Gleichung \eqref{eqn:kirchhoff} kann nun genähert werden und es folgt
\begin{equation}
  U_\text{C}(t) = \frac{1}{\symup{RC}} \int_0^t U \left(t'\right) \, \symup{d}t'.
  \label{eqn:dformel}
\end{equation}
