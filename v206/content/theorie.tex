\section{Theorie}
\label{sec:theorie}
Zwischen zwei Wärmereservoirs mit unterschiedlichen Temperaturen
gibt es einen Wärmestrom der vom wärmeren ins kältere Reservoir führt\cite{Anleitung}.
Mit einer Wärmepumpe kann die Richtung des Wärmestroms umgedreht werden.

\subsection{Funktionsweise einer Wärmepumpe}
In einer Wärmepumpe wird für den Transport der Wärmemengen ein reales Gas mit
hoher Kondensationswärme verwendet. Die Wärmereservoire werden mit Wasser realisiert.
Beim Verdampfen wird die Wärmemenge $Q_2$ aus dem Reservoir 1 aufgenommen,
beschreibbar mit der Verdampfungswärme L pro Gramm.
Die Wärmepumpe ist so konzipiert, dass das Gas beim Druck $p_\text{a}$ und den
Temperaturen $T_2$ verdampft. Das Reservoir wird um diese
Wärmemenge abgekühlt. Das Gas wird in den Kompressor geleistet,
wo es idealisiert adiabatisch komprimiert wird.
Die Temperatur steigt dadurch weiter an, bis es zur Kondensation kommt,
bei der $Q_1$ an das Reservoir 1 abgegeben wird. Wichtig ist der zu $p_\text{a}$ unterschiedliche
Druck $p_\text{b}$ für die Kondensation. Innerhalb der Reservoire
sind Kupferspiralen verbaut um den Wärmeaustausch zu vereinfachen.

\subsection{Güteziffer}
Nach dem ersten Hauptsatz der Thermodynamik folgt
\begin{equation}
      Q_1=Q_2+A\:.
      \label{eqn:thermoeins}
\end{equation}
$Q_1$ ist die Wärmemenge, die an das warme Reservoir abgegeben,
$Q_2$ folglich die Wärmemenge die dem kalten Reservoir entnommen wird.
$A$ ist die Arbeit, welche für den Transport der Wärmemenge benötigt wird.
Die beschreibende Größe einer Wärmepumpe, die Güteziffer $ν$ beschreibt wie gut
der Wärmetransport gelingt, mit
\begin{equation}
      ν = \frac{Q_1}{A}\:.
      \label{eqn:guete1}
\end{equation}
Die reale Wärmepumpe kann den Wärmetransport energetisch nicht umkehren,
bei der idealen Wärmepumpe kann der Prozess reversibel verlaufen.
In Formeln ist die Beziehung in Gleichung \ref{eqn:tempverhaeltnis}
für die ideale \enquote{=} und \enquote{<} bei der realen Wärmepumpe,
\begin{equation}
      0\leq\frac{Q_1}{T_1}-\frac{Q_2}{T_2}\:.
      \label{eqn:tempverhaeltnis}
\end{equation}
Die Güteziffern
\begin{equation}
      ν_\text{real}<ν_\text{id}=\frac{Q_1}{A}=\frac{T_1}{T_1-T_2}
      \label{eqn:guete2}
\end{equation}
zeigen, dass die Effektivität der Wärmepumpe steigt,
wenn die Temperaturdifferenz kleiner ist.

Mit Messwerten kann die Güteziffer $ν$ über die Beziehung
\begin{align}
      ν &= \frac{\increment Q_1}{N \increment t} \\
      &= \frac{m_1 \symup{c}_\text{w}+m_\text{k} \symup{c}_\text{k}}{N}
      \cdot\frac{\increment T_1}{\increment t}
      \label{eqn:guete}
\end{align}
bestimmt werden. $m_1 \symup{c}_\text{w}$ ist die Wärmekapazität des
Wassers im ersten, warmen, Reservoir und $m_\text{k} \symup{c}_\text{k}$
die der Kupferspirale und des Eimers.

Der Massendurchsatz, die transportierte Menge an Gas, kann mit der vom Reservoir 2
abgegebenen Wärmemenge $Q_2$ bestimmt werden. Diese kann mit der Temperaturdifferenz
und den Wärmekapazitäten beschrieben werden, somit folgt bei bekannter Verdampfungswärme L:
\begin{align}
      \frac{\increment m}{\increment t}
      &= \frac{1}{\symup{L}}\cdot\frac{\increment Q_2}{\increment t} \\
      &= \frac{m_2 \symup{c}_\text{w}
      + m_\text{k} \symup{c}_\text{k}}{\symup{L}}\cdot
      \frac{\increment T_2}{\increment t} \:.
      \label{eqn:massendurchsatz}
\end{align}

Für die mechanische Kompressorleistung wird angenommen, dass die Kompression des
Gases adiabatisch erfolgt. Es erfolgt also idealisiert kein Wärmeaustausch mit der
Umgebung. Die Dichte $ρ$ des Gases beim Druck $p_\text{a}$ und der Temperatur $T$ wird mit
\begin{equation}
  \label{eqn:rhö}
      ρ  = \frac{p_\text{a}}{\symup{R}_\text{s} T}
\end{equation}
nach \cite{Gasgleichung} bestimmt,
$\symup{R}_\text{s} = \SI{8.31}{\joule\per\mol\per\kelvin}$
ist die allgemeine Gaskonstante.
Die Formel für $N_\text{mech}$ folgt aus der Poissonschen Gleichung
und der verrichteten Arbeit, nach \cite{Anleitung},
\begin{equation}
      N_\text{mech} = \frac{1}{κ-1}
      \left(p_\text{b} \sqrt[κ]{\frac{p_\text{a}}{p_\text{b}}} - p_\text{a}\right)
      \frac{1}{ρ} \frac{\increment m}{\increment t}\:.
      \label{eqn:leistung}
\end{equation}
$κ$ ist das Verhältnis der Molwärmen bei konstantem Druck $\symup{C}_\text{P}$
und bei konstantem Volumen $\symup{C}_\text{V}$.
