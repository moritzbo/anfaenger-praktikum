\section{Durchführung und Aufbau}
\label{sec:Durchführung}
Um die Leerlaufspannung $U_0$ der Monozelle zu bestimmen, wird diese an einen
Spannungsmesser angeschlossen und mit einer geeigneten Skala die
Leerlaufspannung bestimmt.
\\
\\
Für Teilaufgabe b.) wird mit einer Schaltung wie in Abbildung \ref{fig:schaltung1}
die Klemmenspannung $U_\text{k}$ gegen den Strom $\symbf{I}$ gemessen.
Damit wird später die Leerlaufspannung $U_0$ und der Innenwiderstand
$R_\text{i}$ bestimmt.
An der Stelle von $R_\text{v}$ wird ein Potentiometer verwendet, welches von $0-\SI{50}{\ohm}$
reicht.
\\
\\
Im Anschluss wird mit der Schaltung aus Abbildung \ref{fig:schaltung2}
erneut die Klemmenspannung $U_\text{k}$ gegen
$\symbf{I}$ gemessen. An die Monozelle wird eine Gegenspannung angelegt.
Dabei muss die Gegenspannung jedoch $\SI{2}{\volt}$ größer gewählt werden als die
Leerlaufspannung.
\\
\\
Im folgenden wird die Monozelle gegen einen RC-Generator ausgetauscht und
es wird wie in Teilaufgabe b) verfahren. Anstatt Gleichstrom werden nun der
Rechteck- und Sinusausgang des Funktionengenerators verwendet.
Für die $\SI{0.69}{\volt}$ Rechteckspannung wird ein Potentiometer
$R_\text{a1}$, welches von $20- \SI{250}{\ohm}$ reicht, eingebaut
und es wird wieder der Belastungstrom $\symbf{I}$ gegen die Klemmenspannung
$U_\text{k}$ gemessen.
Bei dem $\SI{1}{\volt}$ Sinusausgang wird ein Widerstand $R_\text{a2}$ mit
einem Variationsbereich von $0.1 - \SI{5}{\kilo\ohm}$ eingebaut und es wird
wieder $\symbf{I}$ in Abhängigkeit von $U_\text{k}$ gemessen.

\begin{figure}
      \caption{Schaltung für die Bestimmung von $U_0$ und $R_\text{i}$.}
      \label{fig:schaltung1}
      \begin{circuitikz}
            \draw
            (0,0) -- (6,0) to[variable resistor, l=$R_\text{v}$] (6,5);
            \draw (2.5,5)  to[ammeter, l=$\si{\ampere}$] (6,5);
            \draw (0,0)    to[battery1=\SI{1.5}{V}] (0,5) -- (2.5,5);
            \draw[<->] (1.5, 5) -- (1.5,0) node[midway, left] {$U_\text{k}$};
            \draw (2.5,0)  to[voltmeter,*-*, l=$\si{\volt}$] (2.5,5);

            \draw (-0.25, 2.2) node{$-$};
            \draw (-0.25, 2.8) node{$+$};
      \end{circuitikz}
\end{figure}

\begin{figure}
      \caption{Schaltung für die Bestimmung der Leistung der Monozelle.}
      \label{fig:schaltung2}
      \begin{circuitikz}
            \draw (0,0) -- (6,0)
                  to[dcvsource] (6,2.5)
                 to[variable resistor, l=$\symup{R}_\text{v}$] (6,5);
                 \draw (2.5,5) to[ammeter, l=$\si{\ampere}$] (6,5);
                 \draw (0,0)   to[battery1=\SI{1.5}{V}]
                 (0,5) -- (2.5,5);
                 \draw (2.5,0) to[voltmeter,*-*, l=$\si{\volt}$] (2.5,5);

                 \draw (-0.25, 2.2) node{$-$};
                 \draw (-0.25, 2.8) node{$+$};

                 \draw (6.4, 0.85) node{$-$};
                 \draw (6.4, 1.65) node{$+$};
           \end{circuitikz}
  \end{figure}
