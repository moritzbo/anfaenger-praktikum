\section{Theorie}
\label{sec:theorie}
\subsection{Fourier'sche Theorem}
Das Fouriersche Theorem besagt, falls die Reihe
\begin{equation}
  f(t) =
  \frac{1}{2} a_0 +
  \sum_{n=1}^\infty
  \left(
  a_n \cos\left(\frac{2\symup{π}n}{\symup{T}}t \right) +
  b_n \sin\left(\frac{2\symup{π}n}{\symup{T}}t \right)
  \right)
  \label{eqn:fouriertheorem}
\end{equation}
gleichmäßig konvergiert, beschreibt sie eine periodische Funktion
\cite{Anleitung} $f(t)$.
Die Koeffizienten $\symup{a}_n$ und $\symup{b}_n$
werden gemäß
\begin{align}
  a_n &= \frac{2}{\symup{T}}
  \int_0^\text{T} f(t)
  \cos\left(\frac{2\symup{π}n}{\symup{T}}t \right) \, \symup{d}t
  \label{eqn:akoeffizient}\\
  b_n &= \frac{2}{\symup{T}}
  \int_0^\text{T} f(t)
  \sin\left(\frac{2\symup{π}n}{\symup{T}}t \right) \, \symup{d}t
  \label{eqn:bkoeffizient}
\end{align}
berechnet \cite{Anleitung}.

\subsection{Fourier-Transformation}
Bei der Fourier-Transformation kann man eine Funktion, ob periodisch oder nicht,
in ihre Fourierkoeffizienten zerlegen. Die Fourier-Transformation berechnet
man gemäß:
\begin{equation}
  g(ν) = \int_{-\infty}^{\infty} f(t) \, \symup{e}^{\symup{i} ν t} \, \symup{d}t
  \label{eqn:fouriertrafo}
\end{equation}
Ist die betrachtete Funktion $f(t)$ nicht periodisch, so ist $g(ν)$ eine nicht
weiter spezifikierbare Funktion. Wenn $f(t)$ periodisch ist, ist $g(ν)$ eine
Reihe von δ-Distributtionen, die, wie die Fourierreihe, konvergiert.
Bei diesem Versuch führt das Oszilloskop die
Fourier-Transformation gemäß \ref{eqn:fouriertrafo} durch und der Betrag von
$g(ν)$ berechnet sich gemäß:
\begin{equation}
  |g| = \sqrt{\symfrak{Re}^2(g)+\symfrak{Im}^2(g)}
  \label{eqn:betrag}
\end{equation}

\subsection{Fourieranalyse}
Bei der Fourieranalyse zerlegt man Schwingungen in ihre Fourierkoeffizienten,
welche somit die ursprüngliche Funktion approximieren.
Da die Approximation im Experiment nur endlich genau ist,
kommt es zu Abweichungen im Schwingungsmuster, welche als Gibbs'sches Phänomen
bezeichnet werden, vgl \ref{sec:gibb}.
Für gerade Funktionen bei denen $f(t) = f(-t)$ gilt, sind die Amplituden
$\symup{b}_n = 0$. Für ungerade Funktionen, das heißt $f(t) = -f(-t)$, werden
alle $\symup{a}_n = 0$.
Da bei der Analyse die Resonanzspannung $\symup{U}_{\text{res}}$ proportional
zu den Amplituden der Oberwellen $\symup{A}_{\text{Oberwelle}}$ steht, können
sämtliche Oberwellen gemäß
\begin{equation}
    \frac{ν_{\text{res}}}{n}
\end{equation}
angeregt werden, wobei $\enquote{n}$ die Anregung der n-ten Oberwelle beschreibt.
Die Resonanzfrequenz ist gemäß
\begin{equation}
  ν_{\text{res}} = \frac{1}{2\symup{π}\sqrt{\symup{LC}}}
\end{equation}
definiert. Dabei ist $\symup{L}$ die Induktivität und $\symup{C}$ die
Kapazität, welche durch die Bauteileigenschaften festgelegt sind. Durch die
Resonanzfrequenz können sehr kleine
Fourierkomponenten bestimmt werden, da bei der Resonanzfrequenz, wie der Name
schon andeutet, das System in Resonanz gebracht wird, und die größtmögliche
Amplitude auftritt.

\subsection{Fouriersynthese}
Bei der Synthese kann eine Schwingung aus ihren errechneten Fourier-Koeffizienten
zusammengesetzt werden.
Dies geschieht mit einem Oberwellengenerator, welcher phasenfeste Schwingungen
generiert, zum Beispiel sinusförmig. Die Frequenzen sind ganzzahlige Vielfache
der Eigenfrequenz ν.

\subsection{Gibbs'sches Phänomen}
\label{sec:gibb}
Als Gibbs'sche Phänomen werden die Überschwingungen bezeichnet,
welche durch Approximation an eine unstetige Stelle $t_0$ einer Funktion
$f(t)$ auftreten. Da das Fourier'sche Theorem eine unendliche Reihe beschreibt,
ist es unvermeidlich diese Abweichungen zu umgehen, wie in
\eqref{fig:rechteck} deutlich zu sehen. Die Anzahl an angelegten
Oberschwingungen kann die Abweichungen verringern.

\subsection{Abtasttheorem}
Bei der Fourieranalyse kann die zu untersuchende Funktion nur durch diskrete
Messpunkte, und nicht kontinuierlich, aufgezeichnet werden.
Wenn die zeitliche Differenz des Werteaufzeichnens viel größer als die
Periodendauer T ist, treten Fehler in der Darstellung der Spannungspeaks auf.
Das Abtasttheorem besagt nun, wenn die Abtastfrequenz $ν_{\text{Abtast}}$
größer ist als das doppelte vorkommende Frequenzmaximum der Funktion $f(t)$,
wird der Fehler vernachlässigbar klein.
\begin{equation}
  ν_{\text{Abtast}} > 2ν_{\text{max}}
  \label{eqn:abtast}
\end{equation}
\section{Fehlerrechnung}
Der Mittelwert eines Datensatzes, mit N Werten, berechnet sich nach
\begin{equation}
  \bar{x} = \frac{1}{N} \sum_{i=0}^{N} x_i.
  \label{eqn:mittelwert}
\end{equation}
\newpage
