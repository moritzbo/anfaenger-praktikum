\section{Diskussion}
\label{sec:Diskussion}
Bei der Fourieranalyse konnte während des Versuches am
Oszilloskop beobachtet werden, dass die jeweils wegfallenden $\symup{a}_n$ oder
$\symup{b}_n$ der Rechteck-, Dreieck- und Sägezahnspannung nicht genau Null
sind. Außerdem konnten die genauen Spannungen nicht präzise genug abgelesen
werden, was zu Ungenauigkeiten führt. \\
Die gemessenen Werte der Rechteckspannung sind, wie in Tabelle
\eqref{tab:rechteckwerteanalyse} zu sehen, sehr gut, bezogen auf die
Theoriewerte.
Die Abweichungen kommen wahrscheinlich daher, dass die Abtastfrequenz des
Oszilloskops zu groß ist um den Fehler vernachlässigbar klein zu machen.
Dieser Fehler macht sich auch bei der Dreieckspannung und bei der
Sägezahnspannung bemerkbar.
Weil die Amplituden der Dreieckspannung mit $\sfrac{1}{\symup{n}^2}$ fallen,
ist es nur schwer realisierbar, Messwerte mit geringem Fehler zu erhalten.
Dies liegt, wie auch bei der Sägezahnspannung, an der auf dem Oszilloskop zu
beobachtenden Schwingung der Amplituden.
\\
\\
Die Fouriersynthese für die Koeffizienten der Rechteckspannung liefert
den Plot \ref{fig:rechteck}, welcher aufgrund der in der Auswertung
aufgeführten Werte nahe an der Theoriekurve liegt. Dieser weißt aber, durch
zum einen Messungenauigkeiten und Einstellungenauigkeiten an den Geräten,
zum anderen die Uneindeutigkeit der Phase bei den Lissajousfiguren,
vor allem aber aufgrund der geringen Anzahl an Oberwellen keine
Achsenparallelen Amplituden auf.

Die zusammengesetzte Funktion aus den Koeffizienten für die Dreieckspannung
\ref{fig:dreieck} ist am besten gelungen, da es nur wenige Oberwellen gab,
welche einen Messfehler beisteuern konnten. Trotz der wenigen Oberwellen kann
hier dennoch sehr deutlich ein dreieckiger Spannungsverlauf erkannt werden.

Bei der Sägezahnspannung \ref{fig:saege} ist anzumerken, dass die abfallende
Flanke fast parallel zur y-Achse verläuft und die ansteigende Flanke nahezu
gerade ist.
