\section{Theorie}
\label{sec:theorie}
Die Biegung der elastischen Stäbe erfolgt durch eine Krafteinwirkung.
Die Spannung, Kraft pro Flächeneinheit, ist eine charakteristische Größe
zur Beschreibung der Phänomene.
Diese wird in Schub-, die zur Oberfläche parallel stehende Kraft, oder
Normalspannung $\symup{σ}$, auch Druck genannt, aufgeteilt.
Der Druck setzt sich aus den Kraftkomponenten senkrecht zur Oberfläche
zusammen.

\subsection{Hookesches Gesetz}
Das Hookesche Gesetz
\begin{equation}
   \symup{σ} = \symup{E} \: \frac{\increment \symup{L}}{\symup{L}}.
   \label{eqn:hookges}
\end{equation}
ist der lineare Zusammenhang zwischen $\symup{σ}$, dem Druck,
und der relativen Änderung des betrachteten Körpers.
Dieses gilt jedoch nur bei kleinen relativen Änderungen
$\increment\symup{L} / \symup{L}$.
E ist der materialspezifische Elastizitätsmodul und in
\eqref{eqn:hookges} die Proportionalitätskonstante.

\subsection{Biegung bei einseitiger Auflage}
Die Biegung ist eine spezielle Form der Deformation, bei
der schon eine kleine Kraft F zu großen
Auslenkungen führen kann. Dies liegt an dem, am Stab,
angreifenden Drehmoment
\begin{equation}
   \symup{M}_{\symup{F}} = \symup{F} (\symup{L}-x),
\end{equation}
welches den Stab aus seiner Ruhelage auslenkt.
Die oberen Schichten des Stabes werden bei dieser Biegung gestreckt und
die unteren gestaucht.
Dies führt zu einer Schicht die in ihrer Ausdehnung bleibt, die neutrale
Faser.
Die endgültige Biegung des Stabes an jeder Stelle x lässt
sich aus dem Gleichgewicht der Drehmomente bestimmen:
\begin{equation}
   \symup{M}_{\symup{σ}} = \symup{M}_{F} \, .
   \label{eqn:drehmoment}
\end{equation}
Wobei $\symup{M}_{\symup{σ}}$ das Drehmoment beschreibt,
welches durch die herrschenden Zug- und Druckspannungen verursacht wird.
Berechnet wird $\symup{M}_{\symup{σ}}$ mit der Normalspannung $\symup{σ}(y)$
und y, dem Abstand des betrachteten Punktes von der neutralen Faser.
Integriert wird über den Querschnitt Q des jeweiligen Stabes,
\begin{equation}
   \symup{M}_\text{σ} =
   \int_{\symup{Q}} \symup{y} \symup{σ}(\symup{y}) \, \symup{d}q.
\end{equation}
$\symup{σ}$ wird hier analog zu \eqref{eqn:hookges} berechnet.
Betrachtet wird aber die kurze Änderung der Länge im Abstand y zur
neutralen Faser.
Demnach gilt
\begin{equation}
  \symup{σ}(y) = \symup{E} \frac{\delta \symup{x}}{\increment \symup{x}},
\end{equation}
wobei
\begin{equation}
  \delta \symup{x} = \symup{y} \frac{\increment \symup{x}}{\symup{R}}
\end{equation}
über die Geometrie des Problems und die Kleinwinkelnäherung hergeleitet
werden kann. Das führt zu
\begin{equation}
  \symup{σ}(\symup{y}) = \symup{E} \: \frac{\symup{y}}{\symup{R}}
\end{equation}
mit
\begin{equation}
  \frac{1}{\symup{R}} \approx \frac{\symup{d^2 D}}{\symup{dx^2}}.
\end{equation}
Setzt man all dieses in Gleichung \eqref{eqn:drehmoment} ein, folgt
\begin{equation}
  \symup{E} \, \frac{\symup{d^2 D}}{\symup{dx^2}}
  \int_{\symup{Q}} y^2 \, \symup{dq}
  = \symup{F}(\symup{L} \: - \: x).
  \label{eqn:drehelast}
\end{equation}
Durch Zuhilfenahme des Flächenträgheitsmoments
\begin{equation}
   \symbf{I} \coloneq \int_{\text{Q}} y^2 \, \symup{d}q(y)
\end{equation}
\label{eqn:flaechentraegheit}
kann diese vereinfacht werden. Für einen
einseitig eingespannten Stab wird somit die folgende Formel benutzt:
\begin{equation}
   \symup{D}(x) = \frac{\symup{F}}{2\symup{E} \symbf{I}}
   \left(\symup{L}x^2 - \frac{x^3}{3} \right)
   \text{\; (für 0 $\leq$ x $\leq$ L).}
   \label{eqn:auflage1seite}
\end{equation}

\subsection{Biegung bei zweiseitiger Auflage}
Um den Elastizitätsmodul für einen zweiseitig eingespannten
Stab zu berechnen, muss der Drehmomentansatz abgeändert werden, sodass
jede Seite betrachtet werden kann.
Zunächst wird der Teil des Stabes mit
\begin{equation*}
  0 \leq x \leq \frac{\symup{L}}{2}
\end{equation*}
betrachtet. Hierfür ist
\begin{equation}
  \symup{M}_{\symup{F}} = - \frac{\symup{F}}{2} x.
\end{equation}
Gleichung \eqref{eqn:drehelast} lässt sich, mit einsetzen und integrieren, nach
\begin{equation}
  \symup{D}(x) = \frac{\symup{F}}
  {48 \, \symup{E} \symbf{I}}
  \left( 3 \symup{L}^2 x - 4 x^3 \right)
  \label{eqn:auflage2seiterechts}
\end{equation}
umformen. Die erste Integrationskonstante wird über die horizontale Steigung
im Mittelpunkt des Stabes bestimmt.
Die zweite wird durch das Festlegen des Nullniveaus,
im Auflagepunkt, eliminiert. Der zweite Teil des Stabes wird mit
\begin{equation*}
  \frac{\symup{L}}{2} \leq x \leq \symup{L}
\end{equation*}
charakterisiert. Das durch die angehängte Masse wirkende Drehmoment ist jetzt
\begin{equation}
  \symup{M}_{\symup{F}} = - \frac{\symup{F}}{2} (\symup{L} - x).
\end{equation}
Mit der Vorgehensweise vom ersten Teil folgt
\begin{equation}
   \symup{D}(x) = \frac{\symup{F}}{48\symup{E}\symbf{I}}
   \left(4x^3 - 12\symup{L}x^2 + 9\symup{L}^2 x - \symup{L}^3\right).
   \label{eqn:auflage2seitelinks}
\end{equation}
